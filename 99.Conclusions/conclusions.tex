\subsection*{Conclusion}

This section brings this thesis to a closure. Here we make a summary and present the most prominent future work directions.

In this thesis, we investigated and validated modelling methodologies to simulate the radio access for a XR conference meeting application. Our main goal was to create a complete framework that allows us to perform sensitivity analysis and that way enable assessments on how deployment and configurations impact application performance. With such insights we can derive deployment guidelines, like the number of antennas, the position and number of the base stations, and spectrum assignments. Additionally, we can create autonomously managed systems that automatically adjust radio-layer configurations, such as scheduling parameters and beamforming algorithms, in order to enhance the service provision given certain available resources and a given channel state.

Naturally, such task requires an extensive background and literature review since the majority of the components we combine into a framework have had plenty of research.

We do this in Chapter 2. Firstly, social virtual reality meeting applications are seen in lights of network requirements. Then we probe what aspects in the application have influence in the radio channel, thus we investigate where cameras, headsets and base stations are placed, and how users move in a meeting. Then, we review traffic models and read on different approaches to optimise the radio layer for this demanding real-time application.

We proceed to survey the key technologies of today. Massive MIMO and mmWaves are the answer, and in the same section we present what they entail. Then we examine how these technologies are standardised and used in today's emerging 5G networks. We also inspect relevant physical layer specifications and network equipment. We conclude the background by selecting the radio channel generator that creates the propagation environment on top of which we simulate transmissions.

Chapter 3 is where we show how everything comes together in our simulation framework. We present all modelling considerations, starting with the application. We describe the physical setting in the room, what antennas are used and how they are placed. We model the head movement of a user and we detail a flexible application traffic model based on video streaming.

Subsequently, we show how the propagation environment changes with the previous application considerations, like antenna placement and user behaviour. The radio channel is measured by the network to make resource management decisions. We present and model the tasks carried by the network. These include channel state information acquisition, user scheduling and the actual transmission.

In Chapter 4 is assessed the result of all modelling considerations. In the previous chapter we defined the parameters that dictate all modelled aspects and in this chapter we give them values to precisely describe the simulation scenarios. We see how the channel quality changed from a conference with a single user to a multi-user conference. More importantly, we verified that our modelling produces realistic results that relate coherently among themselves. We can measure throughputs and latencies and quantify the QoS from a set of application, propagation environment and network settings. This tells us this simulator has been successfully designed.

This work forms a basis and provides tools for further research. 
