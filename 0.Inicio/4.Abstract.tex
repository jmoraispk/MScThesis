\begin{abstract}

    One of the most challenging applications targeted by evolving (beyond-)5G technology is virtual reality (VR). Particularly, ‘Social VR’ applications provide a fully immersive experience and sense of togetherness to users residing at different locations. To support such applications the network must deal with enormous traffic demands, while keeping end-to-end latencies low. Moreover, the radio access network must deal with the volatility and vulnerability of mm-wave radio channels, where even small movements of the users may cause line-of-sight blockage, causing severe throughput reductions and hence Quality of Experience (QoE) degradation or even lead to loss of connectivity. In this work we present an integral modelling approach for feasibility assessment and performance optimisation of the radio access network for Social VR applications in indoor office scenarios. Such modelling enables us to determine the performance impact of e.g. ‘natural’ human behaviour, the positions and configurations of the antennas and different resource management strategies. Insights into these issues are a prerequisite for setting up guidelines for network deployment and configuration as well as for the development of (potentially AI/ML-based) methods for dynamic resource management and tuning of radio access parameters to best support Social VR applications.
    
\end{abstract}