%%%%%%%%%%%%%%%%%%%%%%%%%%%%%%%%%%%%%%%%%%%%%%%%%%%%%%%%%%%%%%%%%%%%%%
% List of symbols
\pdfbookmark[1]{List of Symbols}{los}



% IF there's time:
% MIGRATE THIS LIST OF SYMBOLS TO \newsym based! When the thesis is written, so we know where the first
% instance of the symbol is.

% \newsym{<description>}{<symbol>}
% from: https://personal.utdallas.edu/~kxh060100/symlist.pdf

% ONLY DO THIS if the 'dots between description and page number
% can be removed.

% It would also be nice to have different headings.




% RULE 1:
% NOTE ON THE MATHEMATICAL SYMBOLS TO USE
% As epsilon, use the first variant!

%\varepsilon for ε and \epsilon for ϵ
%\theta for θ and \vartheta for ϑ
%\pi for π and \varpi for or ϖ
%\rho for ρ and \varrho for ϱ
%\sigma for σ and \varsigma for ς
%\varphi for φ and \phi for ϕ

% Start of the list of Symbols
% have 3 lists: Sets, greek letters, normal letters


% RULE 2:
% All symbols are italic (subscripts and superscripts don't need to be)
% except matrices 

% \tab is given in the tabto package
\newcommand\mytab{\tab \hspace{-5cm}}

% If the box is not visually overfull, ignore the warning
% it has to do with the huge tab.



\chapter*{List of Symbols} \todo{should this be called nomenclature instead?}
(not perfectly alphabetically ordered \ii{yet})

\subsubsection*{Latin alphabet}

% A

% B
$B$ \mytab bandwidth \\
$BLER_0$ \mytab target \acs{BLER} \\



% C

% D

% E
$E_b$ \mytab energy per bit \\

% F

% G

% H
$\bm{H}_{bmp}$ \mytab channel matrix between BS $b$ and UE $m$ in BSs polarisation $p$\\

% I
% J
$j$ \mytab imaginary unit $\left(j = \sqrt{-1}\right)$\\


% K
$k_B$ \mytab Boltzmann constant\\

% L for lengths, and latencies -> this will change
$L_{max}$ \mytab maximum latency for the radio link \\
$L_\text{sch}$  \mytab length of scheduling information update period \\
$L_\text{csi}$  \mytab length of \acs{CSI} update period \\
$L_\text{slot}$ \mytab length of slot period \\
$L_\text{TB}$ \mytab length of a transport block, in bits \\

% M

% N (for numbers of something, mainly)
$N_0$ \mytab noise power spectral density \\
$N_r$ \mytab number of receive antennas \\
$N_t$ \mytab number of transmit antennas \\
$N_{phy}$ \mytab number of physical users \\
$N_{vir}$ \mytab number of virtual users \\
$N_\text{CSI Beams}$ \mytab number of beams with CSI-RS, per polarisation \\
$N_\text{BS}$ \mytab number of \acsp{BS} \\
$N_\text{UE}$ \mytab number of \acsp{UE} \\
$N_\text{symb}^\text{PRB}$ \mytab number of symbols per PRB \\
$N_\text{bits}^\text{symb}$ \mytab number of bits per symbol \\
$N_\text{info bits}^\text{symb}$ \mytab number of information bits per symbol \\
$N_\text{bits}^\text{slot}$ \mytab number of bits per slot \\
$N_\text{req PRB}$ \mytab number of requested PRBs by a UE for an UL transmission \\
$N_\text{TB}^\text{slot}$ \mytab number of transport blocks per slot \\
$N_{bml}^\text{PRB}$ \mytab number of PRBs scheduled for a transmission \\
${ }$ \mytab between BS $b$ and UE $m$ in layer $l$ \\
$NF_r$ \mytab noise figure at the receiver \\
$N_x$ \mytab number of antennas along the x-dimension\\
$N_y$ \mytab number of antennas along the y-dimension\\
$NF_\text{BS}$ \mytab BS noise figure \\
$NF_\text{UE}$ \mytab UE noise figure\\

% O

% P (used for powers)
$P_n$ \mytab noise power \\
$P_r$ \mytab received power \\
$P_s$ \mytab received signal power \\
$P_t$ \mytab transmit power \\
$P_t^\text{UE}$ \mytab maximum transmit power per UE\\
$P_t^\text{BS}$ \mytab maximum transmit power per BS \\
$P_{bml}^\text{BS}$ \mytab transmit power at the BS for a link between BS $b$ and UE $m$ in layer $l$ \\
$P_{bml}^\text{UE}$ \mytab transmit power at the UE for a link between BS $b$ and  UE $m$ in layer $l$ \\
$P_\text{IntraCI}$ \mytab interference power from intra-cell interference sources \\
$P_\text{InterCI}$ \mytab interference power from inter-cell interference sources \\
$P_\text{ILI}$ \mytab interference power from inter-layer interference \\
${ }$ \mytab (applicable when there is a multi-layer transmission) \\
$P_\text{PRB}$ \mytab transmit power per PRB \\

% Q
$Q_m$ \mytab modulation order \\
$Q_{P-I}$ \mytab P-frame to I-frame ratio \\

% R
$R_b$ \mytab bit rate \\
$\hat{R}_b$ \mytab estimated bit rate \\
$R_c$ \mytab code rate \\
$\overline{R}_{DL}$ \mytab average application information rate in the \ac{DL}\\

% S
$s_\text{TDD}$ \mytab TDD split \\
$SINR_\text{eff}$ \mytab effective SINR, i.e. aggregated over all scheduled PRBs \\
$SINR_i$ \mytab SINR of the i-th PRB \\
$S_t$ \mytab power spectral density at transmission\\
$S_I$ \mytab size of I frame\\
$S_P$ \mytab size of P frame\\

% T for periods [s] and Temperatures
$T_\text{slot}$  \mytab slot duration \\
$T$ \mytab noise temperature\\

% U

% V

% X

% Y

% W
$\bm{w}$ \mytab vector of beamforming weights \\
$\bm{w}_{\phi, \theta}$ \mytab vector of beamforming weights from a GoB with direction $(\phi, \theta)$ \\
$\bm{w}_{i,j}$ \mytab vector of beamforming weights from a GoB with grid indices $(i,j)$\\

% Z

%\vspace{1cm}

\subsubsection*{Greek alphabet}

% Alpha (Α α)
$\alpha_P$ \mytab power compensation factor for UL power control\\

% Beta (Β β)

% Gamma (Γ γ)
$\gamma_{OLLA}$ \mytab step size for OLLA parameter ($\Delta_{OLLA}$) update\\

% Delta (Δ δ)
$\Delta_{OLLA}$ \mytab outer loop link adaptation step\\

% Ε ε
% Ζ ζ
% Η η
% Θ θ
% Ι ι
% Κ κ
% Λ λ
% Μ μ
% Ν ν
% Ξ ξ
% Ο ο
% Π π
% Ρ ρ
$\rho$ \mytab number of bits per symbol, used in SINR mapping algorithm (MIESM)\\

% Σ σ/ς
% Tau (Τ τ)  - used for delays
$\tau_\text{CSI}$   \mytab \acsp{CSI} delay in number of \acsp{TTI} \\
% Υ υ
% Φ φ
% Χ χ
% Ψ ψ
% Ω ω







\subsubsection*{Sets:}

$\mathbb{N}$ \mytab set of natural numbers \\
$\mathbb{N}_0$ \mytab set of natural numbers including zero\\





\subsubsection*{Other nomenclature}


$\bm{A}$ \mytab matrix\\

$\bm{a}$ \mytab column vector\\

$|\bm{a}|$ \mytab euclidean norm of vector $\bm{a}$\\

$\bm{A}^\text{T}$ \mytab transpose of $\bm{A}$\\
$\bm{A}^\text{H}$ \mytab Hermitian of $\bm{A}$, also know as the transpose conjugate of $\bm{A}$\\

% $\ceil{a}{b}$ \mytab Ceil $a$ to $b$ decimal places, i.e. round $a$ to the next $10^-b$. 
$\ceil{a}$ \mytab ceil $a$, i.e. round up $a$ to the nearest integer\\

$\floor{\cdot}$ \mytab floor $a$, i.e. round down $a$ to the nearest integer \\


% $A_\text{[dB]}$ \mytab the quantity A is expressed in dB.\\






\clearpage
\thispagestyle{empty}


\cleardoublepage
% Pages number is starting now with arabic style... until now it was on roman mode
\pagenumbering{arabic} \setcounter{page}{1}
\baselineskip 18pt