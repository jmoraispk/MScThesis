\pdfbookmark{Acknowledgments}{Acknowledgments}
\begin{acknowledgments} 

This is going to be long, I have plenty of people to thank and little shame about how long it takes. Their contributions amount to more than my effort since I am sure I could not achieve half of this without them. So I dedicate this work to them:

To my advisor in Portugal, Professor António Rodrigues who taught me my first Radio Systems course. I was helped unconditionally when needed, sometimes through the `door of the horse', and allowed as much freedom as a student could wish for.

To my advising team in TNO, Professor Hans van den Berg, Professor Remco Litjens, and normal-person Sjors Braam, for crucial reviews, more than 50 meetings and the countless emails. To this day, \ii{every time} we talk I learn something. Working with you truly humbles me. 

To Remco, again, which was the main responsible for the invaluable opportunity of doing the thesis in TNO. You might have influenced me more than anyone in the last year. Sometimes, most likely by mistake, that influence was positive. Let me just defend this and I will get you your side of the deal for the tens, perhaps hundreds of hours you invested in me. A chocolate bar. Milk, no dark or bitter stuff.

To the people from TNO. I have met more than 60 people from the Networks department alone and I never felt so comfortable around so many incredibly experienced people. During the course of this project, particularly in the Radio Group, I learned and discussed many fascinating new concepts and ideas, some of which allowed me to collaborate in the creation of two patents in other projects within TNO. 

Within TNO, Lucia D'Acunto deserves a paragraph of her own. As I write this you may be swinging on a trapeze at the Dominican Republic like a `youngster', but for many months you were the most important pillar of my balance. I will leave it at that, not to risk getting you more confident than you already are. Thank you.

To Eduardo, Mike, Dinho, Rosa, Lucia (yes, again!) and others, for sharing movement in bouldering, capoeira, parkour and circus stuff, and being the scarce but essential training partners I had during the pandemic.

To Sandra Kizhakkekundil, an intern and now MSc student that I was fortunate to meet and co-advise. Our conversations often help me get clarity on complicated stuff. To Maria Raftapoulou, a PhD student that very directly contributed to this work by running link-level simulations. To Rodrigo Serrão, as you very well said, for raising and holding the bar high.

To my good friends Afonso and Bernardo. For amazing talks that provided the perfect escape after long days. For accountability with the studies and good habits. For the countless rides back and forth when I could not walk. And for the past +10 years of great stories. 

%But especially for proving for more than ten years that real friendship does not look at gender, race or social class. Even though we are all middle-class, white and male... This was a joke. Thanks for so naturally and consistently providing fun times.

\vspace{1cm}

And to my parents. I won't attempt to list why. For everything. Thank you.


\end{acknowledgments}
\clearpage
\thispagestyle{empty}
\cleardoublepage