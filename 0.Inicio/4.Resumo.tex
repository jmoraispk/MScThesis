\begin{resumo}

%One of the most challenging applications targeted by evolving (beyond-)5G technology is virtual reality (VR). Particularly, ‘Social VR’ applications provide a fully immersive experience and sense of togetherness to users residing at different locations. To support such applications the network must deal with enormous traffic demands, while keeping end-to-end latencies low. Moreover, the radio access network must deal with the volatility and vulnerability of mm-wave radio channels, where even small movements of the users may cause line-of-sight blockage, causing severe throughput reductions and hence Quality of Experience (QoE) degradation or even lead to loss of connectivity. In this work we present and validate an integral modelling approach for feasibility assessment and performance optimisation of the radio access network for Social VR applications in indoor office scenarios. Such modelling enables us to determine the performance impact of e.g. ‘natural’ human behaviour, the positions and configurations of the antennas and different resource management strategies. Insights into these issues are a prerequisite for setting up guidelines for network deployment and configuration as well as for the development of (potentially AI/ML-based) methods for dynamic resource management and tuning of radio access parameters to best support Social VR applications.


A quinta geração de comunicações móveis(5G) tornou possíveis serviços inovadores. Em particular, serviços de realidade virtual com componente social oferecem uma experiência totalmente imersiva e uma sensação de união entre usuários. Para suportar tais aplicações, a rede tem que lidar com enormes volumes de tráfego e manter baixas as latências entre os extremos. Além disso, a rede de acesso de rádio deve lidar com a volatilidade e vulnerabilidade dos canais de rádio em ondas milimétricas, onde até mesmo pequenos movimentos dos usuários podem causar bloqueio de linha de vista entre antenas, causando graves reduções de taxa de transferência e, portanto, degradação da qualidade de experiência ou até mesmo perda de conectividade. Neste trabalho, apresentamos e validamos um modelo completo para avaliação e otimização do desempenho da rede de acesso rádio para aplicações de realidade virtual social. Tal dimensionamento permite determinar o impacto no desempenho de factores como o comportamento humano, as posições e configurações das antenas e diferentes estratégias de gestão de recursos rádio. Este conhecimento é imprescindível para definir diretrizes relativas ao equipamento rádio necessário e configuração de rede. Adicionalmente, permite o desenvolvimento de métodos, potencialmente baseados em inteligência artificial, para a gestão dinâmica de recursos e ajuste autónomo de parâmetros no acesso rádio com o intuito de melhor servir utilizadores de realidade virtual.


\end{resumo}
