
\subsection*{Simulate the TTI}

It is time to obtain the outcomes of the realised transmissions: the SINRs each UE got in each PRB and how those resulted in the success or failure of the transport blocks. The steps follow:


\subsubsection*{TBS calculation}

Obtain the \acs{TBS} - we get the number of bits sent in the slot in the numerator in Equation \eqref{eq:bitrate}, and the \ac{TBS} is a fraction of this number. Note that if we have only one \ac{TB} per slot, our bitrate will either be one from the table (if $T_{slot} = 1\text{ms}$), or zero, respectively, if the block has been successfully transmitted or not. Thus, we define the number of transport blocks per slot $N_{TB}^{slot}$ and it related with the size of each \ac{TB} $L_{TB}$ as shown in Equation \eqref{eq:TBS}.


\begin{equation} \label{eq:TBS}
    L_{TB} = \frac{N_{bits}^{slot}}{N_{TB}^{slot}}
\end{equation}

This such manner, $N_{TB}^{slot}$ TBs are sent per slot and the experienced bit rates depend on how many of them are delivered with no errors. 


% constant number of TB influences olla update rate...  

% There are places where the exact tb size is computed. cite where.

\subsubsection*{Compute and Aggregate Realised SINR}
Compute realised SINR per PRB

\subsubsection*{Compute Block Errors}

Compute \ac{BLER} from $SINR_{eff}$ and the MCS used for the transmission by using curve with the correspondent CQI index in Figure \ref{fig:blercurves}.

Flip a \acs{BLER}-biased coin to determine if each block was well received or not.
    

\subsubsection*{Update Link Adaptation, Buffers and Performance Indicators}

we need to update the buffers with the information that was transmitted. We model an ordered buffer, i.e. if only third transport block in a set of five block gets lost, then the bits that would be sent in that transport block are not removed from the buffer.

In essence, this means that the BLER may cause packets to be arrive out of order. This phenomenon is represented in Figure \ref{fig:buf}: assuming a TB with the same size as a packet, the bits that didn't arrive successfully are kept in the transmission buffer.

\image{SLS/ordered_buffer.png}{Buffer state before and after a transmission, with $TBS = L_\text{pck}$}{fig:buf}{.4}


Further note that if a TB is not the size of a packet, which happens more frequently than otherwise, and gets lost, then the bits from more than one packet will be left waiting at the transmission buffer. This increases the likelihood that a packet is discarded due to excessive delay, but makes the modelling considerable more realistic. So, the latencies supported in a given scenario are closer to reality, which, unfortunately, means higher latencies.
