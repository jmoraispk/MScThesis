\section*{Overview}

In this chapter we define all methods and models employed in this work.

Firstly, we present a use-case of an XR conference and the relevant aspects that influence the radio layer. Namely, user disposition, antenna placement, user behaviour and traffic characteristics. Some aspects play a role in the 
capabilities of the network to cope with the requirements, namely the antennas. And some even impact directly the quality of the propagation channel.

Secondly, we have a brief look at how the indoors propagation channel is modelled, how it looks like and how some of the previously defined application-layer parameters influence it. This section brings together the literature review on mmWaves and radio channels models.


Lastly, we present extensive modelling of radio access network aspects. We list the tasks to be carried out by the network in order to simulate a radio transmission. Among these tasks is the acquisition of channel knowledge, resource allocation and result of performing a transmission in a given beam, with so much power, among many other parameters that define each transmission.

In this last section we heavily integrate most radio-related concepts reviewed in the literature review section, as well as the 5G physical layer standardisation. We also present useful tools and frameworks to perform system-level simulations on wireless communications systems.


Moreover, although it should be rare, some quantities will be presented only as parameters without values. We deliberately illustrate, through Figures and Tables, how equations and modelling considerations influence the system. All parameters in this chapter can be changed in order to simulate different scenarios. As such, their values are only attributed in the results section.