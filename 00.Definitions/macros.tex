\newcommand{\defining}[1]{
	\bb{Definition}\hspace{-.2cm} \ii{#1}:
}

\newcommand{\bb}[1]{
	\textbf{#1}
}
 
\newcommand{\ii}[1]{
	\textit{#1}
}

\newcommand{\ul}[1]{
	\uline{#1}
}


\newcommand{\image}[4]{
	\begin{center}

		\includegraphics[scale = #4]{#1}
    	\captionsetup{type=figure} 
   		\caption{#2}			
        \label{#3}
        
    \end{center}
}
%possible to use: \ifthenelse{\isempty{#1}}{#1}{default value for #1} for optional arguments

\newcommand{\imagenolabel}[3]{
	\begin{center}

		\includegraphics[scale = #3]{#1}
		\captionsetup{type=figure} 
		\caption{#2}			
		
	\end{center}
}



%to make an \hrule have the ruler in the middle of the line. Use \Vhrulefill
\def\Vhrulefill{\leavevmode\leaders\hrule height 0.7ex depth \dimexpr0.4pt-0.7ex\hfill\kern0pt}


\newcommand{\quickimage}[2]{
	\begin{center}
		\includegraphics[scale = #2]{#1}
    \end{center}
}

\newcommand{\quickimagesidebyside}[4]{
	\begin{figure}[!h]
		\centering
		\begin{subfigure}{0.5\textwidth}
		  \centering
		  \includegraphics[width=#2\linewidth]{#1}
		\end{subfigure}%
		\begin{subfigure}{0.5\textwidth}
		  \centering
		  \includegraphics[width=#4\linewidth]{#3}
		\end{subfigure}
	\end{figure}
}

%Note: you can't use numbers on definitions

%One label on the overall figure
\newcommand{\imagesidebysideonelabel}[8]{
	\begin{figure}[!h]
		\centering
		\begin{subfigure}{0.5\textwidth}
		  \imagenolabel{#1}{#2}{#3}
		\end{subfigure}%
		\begin{subfigure}{0.5\textwidth}
		  \imagenolabel{#4}{#5}{#6}
		\end{subfigure}
		\caption{#7}
		\label{#8}
	\end{figure}
}


%Labels on both subfigures
\newcommand{\imagesidebysidetwolabels}[9]{
	\begin{figure}[!h]
		\centering
		\begin{subfigure}{0.5\textwidth}
		  \image{#1}{#2}{#3}{#4}
		\end{subfigure}%
		\begin{subfigure}{0.5\textwidth}
		  \image{#5}{#6}{#7}{#8}
		\end{subfigure}
		\caption{#9}
	\end{figure}
}


\newcommand{\imagesidebysidenosubfigure}[8]{
	\begin{figure}
		\centering
		\begin{minipage}{.5\textwidth}
			\image{#1}{#2}{#3}{#4}
		\end{minipage}%
		\begin{minipage}{.5\textwidth}
			\image{#5}{#6}{#7}{#8}
		\end{minipage}
	\end{figure}
}




% Game has changed! More than 9 Arguments now!!!

% LIKE THIS:
\newcommand\foo[9]{
    \def\tempa{#1}
    \def\tempb{#2}
    \def\tempc{#3}
    \def\tempd{#4}
    \def\tempe{#5}
    \def\tempf{#6}
    \def\tempg{#7}
    \def\temph{#8}
    \def\tempi{#9}
    \foocontinued
}

\newcommand\foocontinued[7]{%
	% Do whatever you want with your 9+7 arguments here.
	\tempa 
	\tempb
	This is the 10th argument: #1
	This is the 16th argument: #7
}

% How to use: you now use \foo{1}{2}...{9}{10} or until 18, and keep doing it.
% NOTE 1: you have to call \foo at least with 10 arguments, even if \foocontinued can take more, the others will be considered to be empty. Call: \foo{1}{2}{3}{4}{5}{6}{7}{8}{9}{10} to test.
% NOTE 2: that all variables can only be accessed in the last command.
% NOTE 3: latex variables can't have numbers... so 'tempa' is the best we can do.

\newcommand{\imagesidebysidecomplete}[9]{
	\def\tempa{#1}
    \def\tempb{#2}
    \def\tempc{#3}
    \def\tempd{#4}
    \def\tempe{#5}
    \def\tempf{#6}
    \def\tempg{#7}
    \def\temph{#8}
    \def\tempi{#9}
    \imagesidebysidecompletecontinued
}

\newcommand{\imagesidebysidecompletecontinued}[3]{
	\begin{figure}[!h]
		\centering
		\begin{subfigure}{#2\textwidth}
		  \image{\tempa}{\tempb}{\tempc}{\tempd}
		\end{subfigure}%
		\begin{subfigure}{#3\textwidth}
		  \image{\tempe}{\tempf}{\tempg}{\temph}
		\end{subfigure}
		\caption{\tempi}
		\label{#1}
	\end{figure}
}





\newcommand{\threeimagessidebyside}[9]{
	\def\tempa{#1}
    \def\tempb{#2}
    \def\tempc{#3}
    \def\tempd{#4}
    \def\tempe{#5}
    \def\tempf{#6}
    \def\tempg{#7}
    \def\temph{#8}
    \def\tempi{#9}
    \threeimagessidebysidecontinued
}

\newcommand{\threeimagessidebysidecontinued}[9]{
\begin{figure}[h]
    \centering
    \begin{subfigure}[b]{#6\textwidth}
        \image{\tempa}{\tempb}{\tempc}{\tempd}
    \end{subfigure}
     %add desired spacing between images, e. g. ~, \quad, \qquad, \hfill etc. 
      %(or a blank line to force the subfigure onto a new line)
	\begin{subfigure}[b]{#7\textwidth}
		\image{\tempe}{\tempf}{\tempg}{\temph}
    \end{subfigure}
     %add desired spacing between images, e. g. ~, \quad, \qquad, \hfill etc. 
    %(or a blank line to force the subfigure onto a new line)
	\begin{subfigure}[b]{#8\textwidth}
		\image{\tempi}{#1}{#2}{#3}
    \end{subfigure}
    \caption{#4}\label{#5}
\end{figure}
}