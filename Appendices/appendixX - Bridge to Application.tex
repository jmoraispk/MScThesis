\fancychapter{I-frame from application parameters}
\label{sec:beam_steering}


From the application parameters - Requires many application layer parameters: camera resolution (bits per frame), pixel format (channels per pixel), pixel depth(bits per channel), depth resolution ( bits per channel) (note that the cameras are RGB-D cameras, also record depth information), compression ratio, and throughput reduction techniques like tile prediction.
All this is taken into account in equation \eqref{eq:app_layer_param1}. 

\begin{align}
    I_{{\text{size}}_\text{uncomp.}} = \ & \text{camera resolution} \times \nonumber \\
    & \times \left[  \text{pixel format} \cdot \text{pixel depth} + 1 \cdot \text{depth resolution} \right]  \label{eq:app_layer_param1}
\end{align}

And the size of the I frame, pos-compression, is given by Equation \eqref{eq:app_layer_param2}. In this equation, RCR stands for the Remaining Compression Ratio, and symbolises the compression that is left after taking into account the reduction from using P frames. This needs to be done because the video encoding algorithm creates this frames also, therefore the amount of compression given by total video compression has a component of IP frame creation and some remaining compression component, which is the compression we want to apply to the I frame since the IP component we are naturally already applying by considering I and P frames.
\begin{gather}
    I_{size} = I_{{\text{size}}_\text{uncomp.}} \times \text{RCR}  \label{eq:app_layer_param2} \\
    \text{RCR} = \frac{\text{Total Compression Ratio}}{\text{IP Compression Ratio}} \label{eq:app_layer_param3} 
\end{gather}
\begin{align}
    \text{IP Compression Ratio} & = \ \frac{\text{Avg. Frame Size after IP compression}}{\text{Avg. Frame Size before IP compression}} \nonumber \\
    & = \ \frac{I_\text{size}}{I_\text{size} \frac{1 + \text{IP}_\text{ratio} \cdot (GOP - 1)}{GOP}} = \frac{GOP}{1 + \text{IP}_\text{ratio} \cdot (GOP - 1)}\label{eq:app_layer_param4} 
\end{align}

Therefore, bringing all together to a final equation:
\begin{equation}
    I_{size} = \frac{I_{{\text{size}}_\text{uncomp.}} \times \text{Total Compression Ratio} \times GOP}{1 + \text{IP}_\text{ratio} \cdot (GOP - 1)}
\end{equation}

\bb{Note}: a better nomenclature would be $\text{CR}_\text{T}$, $\text{CR}_\text{R}$, $\text{CR}_\text{IP}$, respectively for the total, remaining and IP-attributed Compression Ratios.
