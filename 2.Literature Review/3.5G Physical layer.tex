\section{5G New Radio Physical Layer}
\label{sec:5gphy}


5G \ac{NR} is a new \ac{RAT} developed by \ac{3GPP} for the fifth generation mobile communications network. In this section we present all the relevant radio access physical layer aspects in order to align our modelling decisions compatibly with the industry standards.

According to Qualcomm \cite{qualcomm_innovative_five}, the biggest 5G-compatible smartphone chips manufacturer,the five wireless inventions that define the global 5G standard are:

\begin{enumerate}
    \item Scalable \acs{OFDM} numerology with variable subcarrier spacing;
    \item Flexible slot-based framework;
    \item Advanced \ac{LDPC} channel coding;
    \item Massive MIMO;
    \item Mobile mmWave;
\end{enumerate}

All inventions play a crucial role in the radio access process. Therefore, let us organise this section according to the list above. 


\subsection*{Scalable OFDM}

\ac{OFDM} is a digital multi-carrier modulation scheme. Rather than transmitting a high-rate stream of data with a single carrier, \ac{OFDM} makes use of a large number of closely spaced orthogonal subcarriers that are transmitted in parallel. Each subcarrier is modulated with a conventional digital modulation scheme, such as \ac{PSK} or \ac{QAM}. 

Figure \ref{fig:ofdm} shows an example of OFDM waveforms. Note how the subcarrier nulls line with the peaks of neighbouring subcarriers. If each subcarrier is sampled at its peak, no trace of other subcarriers can be found, i.e. there is no interference.

\imagecapcontrol{Literature Review/ofdm.png}{Example of \ac{OFDM}-based system represented in time and frequency \cite{keysight_concepts_ofdm}}{fig:ofdm}{.45}{-0mm}


In the new air interface, \ac{OFDM} is used for multiple access separating users in different time-frequency resources. To achieve this feat, resources are divided in blocks called \acp{PRB}.  The symbol duration is equal to the inverse of the subcarrier spacing plus roughly 7\% of this value for guard time. The duration of the PRB is equivalent to the duration of a slot and this quantity is often called the \ac{TTI} because it is the minimum division in time to perform operations. 

The values for subcarrier spacing/width and PRB duration in 4G are 15 kHz and 1 ms, respectively. In 5G however the subcarrier spacing value is variable. As such, it is possible to use larger subcarriers to achieve quicker transmissions. The indices for given subcarrier spacing values are called numerologies and Table \ref{tab:num} shows the different  numerologies and the respective subcarrier spacings and slot durations. 

% if possible, include this formula somewhere. $ \Delta f = 2^\mu \times 15 $ kHz

\begin{table}[h]
    \centering
    \caption{Numerology influence on subcarrier spacing and slot duration.}
    \label{tab:num}
    \begin{tabular}{|c|c|c|}
    \hline
    Numerology & Subcarrier Spacing {[}kHz{]} & Slot duration {[}$\mu s${]} \\ \hline
    0          & 15                           & 1000                        \\ \hline
    1          & 30                           & 500                         \\ \hline
    2          & 60                           & 250                         \\ \hline
    3          & 120                          & 125                         \\ \hline
    4          & 240                          & 62.5                        \\ \hline
    \end{tabular}
\end{table}


\subsection*{Slot-based Framework}

The coordination of \ac{UL} and \ac{DL} transmissions is dependent on the slot structure. However, differently from 4G, this structure can change at the end of each transmission period, and the duration of this transmission period can also change \cite{3gpp_slot_periodicity}. Figure \ref{fig:slot_struct} shows how the slot structure is defined in terms of DL slots, the composition of the transition slot in DL, guard and UL transition symbols, and UL slots at the end.

\imagecapcontrol{Literature Review/slot_structure.png}{Slot structure.}{fig:slot_struct}{.45}{-0mm}

Moreover, each slot can have numerous different formats where each symbol is either used for UL, DL or in a \ac{F} manner. \ac{DL}/\ac{UL} slots are used for carrying \ac{DL}/\ac{UL} or \ac{F} symbols. Figure \ref{fig:slot_format} represents two of the least creative but most used options, however practically any combination of DL, UL and F symbols is available in the standards \cite{3gpp_slot_periodicity}. 

\imagecapcontrol{Literature Review/slot_format2.png}{Slot format.}{fig:slot_format}{.28}{-0mm}

On the bottom of Figure \ref{fig:slot_format} is a format that can be used for self-contained slots, where the F symbol is used as a guard interval. Self-contained slots are meant to enable a quick ACK/NACK feedback in low-latency communications, or for massive MIMO UL pilots \cite{qualcomm_innovative_five}. The guard symbols between UL and DL are needed for synchronisation and can their numbers can be smaller for shorter distances. In indoor applications, it is sufficient to use a single guard symbol \cite{rel15_self_contained_slot_qualcomm}.

In essence, the standards allow DL and UL assignment to change on a symbol level, thus making transmissions more efficient since resource needs can be met more precisely, with less resource waste by excess \cite{air_interface_system_principles}.


\subsection*{Modulation and Coding}

We have so far surveyed how transmissions are organised in time and frequency. Now we review the different ways of coding and modulating the data streams into transmissions. The \ac{MCS} influences the rate at which one can convey information.

Streams of data are organised in groups of bits to be encoded into symbols. More bits are sent per symbol with higher modulation orders. Figure \ref{fig:constellation} shows the \ac{IQ} plane constellation diagrams for \ac{BPSK}, \ac{QPSK} and 16-\ac{QAM}. 5G additionally supports 64-QAM and 256-QAM. Each symbol encodes $\log_2(M)$ bits, where $M$ is the modulation order - respectively, 1, 2 and 4 for the modulations in the figure.

\threeimagessidebyside{Literature Review/BPSK.png}{BPSK}{fig:bpsk}{.15}{Literature Review/QPSK.png}{QPSK}{fig:qpsk}{.15}{Literature Review/16QAM.png}{16-QAM}{fig:16qam}{.15}{Constellation diagrams for common digital modulations. \cite{splash_constellations}}{fig:constellation}{0.3}{0.3}{0.3}

More concretely, different symbols have different phases (e.g. Figure \ref{fig:qpsk}) or a combination of different amplitudes and phases (e.g. Figure \ref{fig:16qam}) where the distance to the origin marks the amplitude and the complex argument holds the phase. Moreover, the receiver will perceive a different symbol, i.e. in a slightly different place in the \ac{IQ} plane, than it was transmitted (e.g. due to noise). Thus, the closer symbols in the \ac{IQ} plane supposedly are, the more likely they are to be mistaken by other symbols, and we see that symbols get closer with higher modulation orders.

Several combinations of modulation and coding schemes are available. However, it is required to choose the appropriate scheme that best fits the quality of the channel. Table \ref{tab:mcs} holds the \acs{MCS} that correspond to a certain \ac{CQI} that is reported when the channel state is measured to select what \ac{MCS} to use. Moreover, the table contains the transmitted number of information bits (i.e. excluding coding) per symbol in the `Efficiency' column.

\begin{table}[h]
    \centering
    \caption{\ac{CQI} Table 5.2.2.1-3 from \cite{3gpp-codebooks}}
    \label{tab:mcs}
    \begin{tabular}{|c|c|c|c|}
    \hline
    {\textbf{CQI index}} & {\textbf{Modulation}} & { \textbf{Code rate x 1024}} & {\textbf{Efficiency}} \\ \hline
    0                                      & \multicolumn{3}{c|}{out of range}                                                                         \\ \hline
    1  & QPSK   & 78  & 0.152  \\ \hline
    2  & QPSK   & 193 & 0.377   \\ \hline
    3  & QPSK   & 449 & 0.877   \\ \hline
    4  & 16QAM  & 378 & 1.477 \\ \hline
    5  & 16QAM  & 490 & 1.914 \\ \hline
    6  & 16QAM  & 616 & 2.406 \\ \hline
    7  & 64QAM  & 466 & 2.731 \\ \hline
    8  & 64QAM  & 567 & 3.322 \\ \hline
    9  & 64QAM  & 666 & 3.902 \\ \hline
    10 & 64QAM  & 772 & 4.523 \\ \hline
    11 & 64QAM  & 873 & 5.115 \\ \hline
    12 & 256QAM & 711 & 5.555 \\ \hline
    13 & 256QAM & 797 & 6.227 \\ \hline
    14 & 256QAM & 885 & 6.914 \\ \hline
    15 & 256QAM & 948 & 7.406 \\ \hline
    \end{tabular}
\end{table}


\begin{comment}

\begin{table}[h]
    \centering
    \caption{\ac{CQI} Table 5.2.2.1-3 from \cite{3gpp-codebooks}}
    \label{tab:mcs}
    \begin{tabular}{|c|c|c|c|c|}
\hline
CQI & Bits per symbol & Code Rate x 1024 & Efficiency & Bits per PRB \\ \hline
0   & 0               & \multicolumn{3}{c|}{out of range}            \\ \hline
1   & 2               & 78               & 0.152      & 25.59        \\ \hline
2   & 2               & 193              & 0.377      & 63.33        \\ \hline
3   & 2               & 449              & 0.877      & 147.33       \\ \hline
4   & 4               & 378              & 1.477      & 248.06       \\ \hline
5   & 4               & 490              & 1.914      & 321.56       \\ \hline
6   & 4               & 616              & 2.406      & 404.25       \\ \hline
7   & 6               & 466              & 2.730      & 458.72       \\ \hline
8   & 6               & 567              & 3.322      & 558.14       \\ \hline
9   & 6               & 666              & 3.902      & 655.59       \\ \hline
10  & 6               & 772              & 4.523      & 759.94       \\ \hline
11  & 6               & 873              & 5.115      & 859.36       \\ \hline
12  & 8               & 711              & 5.555      & 933.19       \\ \hline
13  & 8               & 797              & 6.227      & 1046.06      \\ \hline
14  & 8               & 885              & 6.914      & 1161.56      \\ \hline
15  & 8               & 948              & 7.406      & 1244.25      \\ \hline
\end{tabular}
\end{table}


\end{comment}


We have addressed multi-layer transmissions previously, however there are constraints regarding how those layers are coded and modulated. As such, we need to define more carefully terms such as \ac{QoS} flow, codewords and layers.

The \ac{PDSCH} is the main downlink data bearing channel and is allocated to users on a dynamic and opportunistic basis. The \ac{PUSCH} is its \ac{UL} equivalent. Such shared channels carry data in \acp{TB} which correspond to a \ac{MAC} layer \ac{PDU}. They are passed from the \ac{MAC} layer to the \ac{PHY} layer once per \ac{TTI}.

The \acp{TB} from the \ac{QoS} flow have a \ac{CRC} added to them. Then they are segmented into smaller chunks called \acp{CB} and each of those chunks has its own \ac{CRC}. These \acp{CB} are then coded with \ac{LDPC} which is a code that outperforms polar and turbo codes achieving higher code efficiency at considerable lower complexity and with less implementations challenges \cite{qualcomm_innovative_five}. Finally, the \acp{CB} are put back together. Figure \ref{fig:tb_to_codewords} represents the transformation from \acp{TB} to codewords described in this paragraph.

\imagecapcontrol{Literature Review/TB_to_codewords.png}{Block diagram of channel coding chain at the transmitter.}{fig:tb_to_codewords}{.46}{-1mm}

A codeword is scrambled before it is before modulated so as to give the data to be transmitted useful properties that facilitate demodulation. After modulation, each codeword is separated into a maximum of four layers. Then each layer is precoded. Frequently, layers are called independent data streams therefore attention to context is required.

Posteriorly, the precoded signals are mapped to the respective \acp{RE}. Namely reference signals needs require this process such that the receiver knows where that reference signal can be found. Each \ac{RE} consists of one subcarrier used for the duration of one \ac{OFDM} symbol. Given a PRB, by definition, is the use of 12 subcarriers for the duration of 14 OFDM symbols, each PRB consists of 168 \acp{RE}.

Then, \ac{OFDM} symbols are generated and mapped from the logical antenna ports to the physical antenna ports. Such mapping depends on the manufacturer's antenna geometry. We review these considerations in more detail the next section.

Figure \ref{fig:codewords_to_ant} summarises the chain of events from codewords to layers to actual transmission in physical antennas. Although both Figure \ref{fig:tb_to_codewords} and \ref{fig:codewords_to_ant} have transmitter-side block diagrams and descriptions, the receive side is the same, only reversed.

\imagecapcontrol{Literature Review/codewords.png}{Block diagram at the transmitter: from codewords to physical antennas.}{fig:codewords_to_ant}{.37}{-2mm}


There can be a maximum of two codewords per user. The maximum number of transmitted layers also has a limit at eight layers per user when the user is served alone (SU-MIMO) and two layers per user when more than one user is served simultaneously (MU-MIMO) \cite{3gpp-codebooks}.



\subsection*{Massive MIMO}

As mentioned, 5G heavily relies on massive MIMO, and thus on beamforming, to increase its spectral efficiency. Therefore, the performance massive MIMO systems can achieve in 5G is closely related to how beams are created.

There are essentially two types of beams in 5G: access beams, for the initial access to the network, these are broad beams designed to cover large sections of the cell; and traffic beams, which are more directive beams made to serve \acsp{UE}.

The access beams, or \ac{SSB} beams, carry synchronisation signals and the cell \acs{ID} \cite{8407094}. The access procedure is based on beam sweeping where all the SSB beams in a \ac{GoB} are used, one at a time, and the UE chooses the cell of the best received beam. Given that we intend to analyse application performance, there could be cases where access is important, e.g. when the connection breaks and a new access is required. However, we want to avoid loss of connection since it would reflect in major performance dips. We focus on managing the already established traffic beam such that connectivity is never lost.

After the access, \ac{CSI} is required to manage the connection with UEs. 5G's \ac{CSI} framework holds two major phases \cite{beam_management}. The first is the beam management phase where the beams to be used by the \ac{BS} and \ac{UE} are derived. The second is the CSI acquisition phase where, for instance, \acs{CQI} are reported and the \acsp{MCS} are chosen. 

\subsubsection*{Beam Management}

Beam management consists of tuning the beams used to serve UEs in a way that channel quality improves, often consisting of optimising beam direction and increasing its directivity. There is a clear trade-off between beam directivity and robustness since more directive narrower beams can also fall out of alignment more easily causing significant decreases in signal quality or even resulting in loss of connectivity. Beam alignment requires accurate beam tracking based on channel knowledge.

There are two ways of performing beam alignment in \ac{5G}, based on different \acp{RS} \cite{beam_management}: 


\begin{itemize}
    \item \ac{CSI-RS} - downlink \ac{RS}. Each RS is sent in UE-specific time-frequency resources and beamformed in a specific direction, as shown in Figure \ref{fig:csirs}. Upon reception, the UE reports the IDs, and optionally the powers, of up to the four best received CSI-RSs, in a feedback quantity called \ac{CRI}. To send the report, the UE transmits with the optimal beam to received the best CSI-RS, unless it had been previously instructed by the BS to use another beam. Alternatively, the UE can feedback a \ac{PMI} suggesting a beam from the GoB based on the measurements performed on the received CSI-RSs;
    \item \ac{SRS} - uplink \ac{RS}. Each RS also uses well-defined \acsp{RE}, but contrary to the \acs{CSI-RS}, each antenna sends one, see Figure \ref{fig:srs}. Upon reception, the BS derives a beam to transmit to the UE, which then either derives the best beam for reception and uses it for transmission too, or uses the beam instructed by the BS \cite{DAHLMAN2018};
\end{itemize}

\imagesidebysidecomplete{Literature Review/csi-rs_beams.jpg}{BS transmitting CSI-RSs \cite{csi_resources_fig}}{fig:csirs}{.8}{Literature Review/srs3.png}{UE transmitting SRSs}{fig:srs}{.3}{Transmission of reference signals for beam management.}{fig:rs_transmission}{.6}{.4}

To bridge the gap to the two types of beamforming defined in Section \ref{sec:types_of_beamforming}: codebook-based beamforming mentioned is CSI-RS-based, while the non-codebook-based beamforming is SRS-based. Naturally, since each is based on a different reference signal and involve different procedures, they effectively result in different beams, exactly like the two types of beamforming do. Note that exactly as feedback-based beamforming, beam management with CSI-RS is based on partial channel knowledge, deriving information only about the best beam. Conversely, SRS-based beamforming, much like free-format beamforming, requires full channel knowledge, i.e. the channel responses from the \ac{UE} antennas.

\subsubsection*{Channel State Information Acquisition}

The CSI acquisition is the same for both cases of beam management. It consists of the BS sending a CSI-RS, the UE deriving CSI and reporting it back to the BS. However, using beam management that is CSI-RS based, both procedures can be compressed in one, and the UE reports the best beam IDs in the CSI feedback.

As seen, beam management and CSI acquisition procedures are not instantaneous nor can be performed in a single time slot. Therefore, there is a delay between CSI measurements and usage. For example, a beam is derived in certain conditions but it is used a few moments later, and the conditions may have changed. 

There is a good reason why CSI-RS is used for acquiring CSI but SRSs are not. The SRSs needs to be transmitted in orthogonal sequences across all UEs such that they do not interfere with one another at the BS. The CSI-RS, on the other hand, only has to be orthogonal across the UEs that could receive that CSI-RS in that RE, and there may be a single UE in that list. Therefore, more CSI-RSs can be sent per UE since they require less orthogonality. More precisely, 32 CSI-RSs can be scheduled for a UE, but only four SRSs can be sent in the uplink by each UE, and the flexibility of using CSI-RSs is greater due to less orthogonality constraints \cite{DAHLMAN2018}. Also, for the case of interference measurements, they can only be measured at the receiving end.

In essence, massive MIMO works with both types of beamforming and the standards also allow both. The academia has supported SRS-based beamforming, while industry has been waging primarily CSI-RS-based \cite{tddVSfdd_massiveMIMO}, judging also by the 3GPP standards \cite{3gpp-codebooks}. And contrary to the academia traditional massive MIMO formulations of having UEs with a single antenna and focusing all complexity at the BS side, the tendency is the growth of antenna arrays in \acs{UE} as well. An example is the modules of 64 dual-polarised antenna elements for mmWave communications in the latest 5G smartphone chips \cite{qualcomm_64_antenna}. And such high number of antenna elements makes channel sounding impractical, thus rendering reciprocity-based massive MIMO less likely.

The uplink of an \ac{SRS} allows the BS to have information on up to four distinct channels to the UE, because up to four orthogonal SRS sequences are available per UE. Therefore, the UE may use up to four antenna ports, each antenna port mapping an SRS sequence to a set of physical antennas. This means that if the UE has four or less antennas, it can send a different SRS per physical antenna and the BS may estimate the channel to every single antenna of the UE. However, when the number of antennas on the UE is greater than four, it will not be possible for the BS to know the transformation that occurred between each UE antenna and each of its antennas. Many authors \cite{7504159} consider an infinite amount of orthogonal SRSs available to be sent by each UE. However, when there is a limited amount of orthogonal sequences allowed per UE, the spatial filter applied to map one SRS to several physical antennas will be part of the transformation on the SRS, and since the BS is only be capable of inverting the complete transformation that occurred on the reference signal, it cannot target individual antennas. As a result, the performance of reciprocity-based massive MIMO will fall short the asymptotic results in academia.
% cite more authors that mention 5G but then consider digital beamforming with no fears...

The interference that results from excessive uplink of SRSs goes by the name of pilot contamination. Although there are some strategies to mitigate it \cite{8094949, 7996674}, they do not work well with so many UE antennas. Moreover, massive MIMO is supported both in \ac{TDD} and \ac{FDD} modes \cite{8255833}, but in \ac{FDD} only feedback-based works due to the lack of reciprocity of having \ac{UL} and \ac{DL} in different frequencies. Therefore, considering the standards, massive MIMO with a GoB seems to be the most promising direction.

%And several flavours of GoB are available: per subcarrier or subband versus whole band; and individual set of beams per UE versus common GoB \cite{tddVSfdd_massiveMIMO}.


\subsection*{Mobile mmWave}

We overviewed how beam-tracking works in 5G, which will play a crucial role in supporting mobility. It becomes an harder challenge in mmWaves where antenna dimensions make it easier to have high number of antennas, thus allowing for much more directive beams. Let us see what are the standardised differences between mmWave and lower frequencies. 

There are two \acp{FR} in 5G. The specific frequencies of each range are represented in Table \ref{tab:frs}, but not all numerologies are available in all bands. \ac{TDD} and \ac{FDD} operation are band-dependent as well. 

\begin{table}[h]
    \centering
    \caption{Differences between Frequency Ranges in 5G.}
    \label{tab:frs}
    \begin{tabular}{c|c|c|}
        \cline{2-3}
        \multicolumn{1}{l|}{}              & Frequency Range 1 & Frequency Range 2 \\ \hline
        \multicolumn{1}{|c|}{Frequencies}  & 410 - 7125 MHz    & 24250 - 52600 MHz \\ \hline
        \multicolumn{1}{|c|}{Numerologies} & 0, 1, 2        & 2, 3           \\ \hline
        \multicolumn{1}{|c|}{Duplexing}    & TDD, FDD       & TDD               \\ \hline
        \multicolumn{1}{|c|}{Overhead}    & UL: 0.08 / DL: 0.14      & UL: 0.10 / DL: 0.18               \\ \hline
    \end{tabular}
\end{table}

%The maximum transmission bandwidth for one \ac{CC} \cite{} 

% The maximum number of \acp{PRB} is 275 which corresponds to 50 MHz, 100 Mhz, 200 Mhz and 400 Mhz in numerologies 0 to 3. 
%Considerably more than 20 MHz in LTE. %cite..


Although \ac{FR}1 is commonly called `sub-6 GHz', the upper limit has been increased to around 7 GHz \cite{3gpp_bands}. Additionally, Table \ref{tab:frs} contains the signalling overheads of each frequency range for UL and DL, in accordance with 3GPP \cite{3gpp_overheads}. One reason overhead in mmWave are bigger has to do with more feedback needed for beam management and CSI acquisition since mmWave channels change faster.


% difference between throughputs and bitrates. Frequency ranges overheads.
% Between the actual application bit rate and the bit rate supplied to layer 3 (packet-level) there are still IP headers overhead, TCP overhead

In FR1 the standard allows practically all frequency bands. In FR2 however, there are more specific bands. Roughly speaking, portions of spectrum are currently considered in the standards: from 24 to 30 GHz and from 37 to 43 GHz, depending on the location around the globe. In Europe, the assigned mmWave band is 24.25 to 27.5 GHz \cite{5g_2020}.
