\section{Contributions}
\label{sec:contributions}

As any research work, the objective of this project is to evolve the state-of-the-art. Here we present what contributions this work makes to current research in the area.

This thesis focuses on radio access challenges for supporting indoor Social XR applications. We present and test a modelling framework covering all relevant aspects needed for feasibility assessments and performance optimisation.

More concretely, and in an orderly manner, our contributions are:

\begin{itemize}
    \item Integral modelling framework comprising a XR conference use case, traffic characteristics, network deployment including spectrum assignment and antennas, propagation environment and the key 5G traffic handling and resource management mechanisms. The framework fills modelling gaps in literature, namely with a head movement model and an application traffic model based on video streaming;
    \item An integration of the developed models in a system-level simulator, which enables extensive sensitivity analysis on antenna deployments, selection of frequency bands, MIMO algorithms, packet scheduling strategies, as well as application use case aspects such as the number of physical and virtual participants, their behaviour in terms of movement, among other;
    \item Validation of the modelling framework through extensive simulations. We prove the modelling considerations generate realistic and coherent results.
\end{itemize}

Insights on how configurations impact performance can apply to different use cases and applications. It is so because application-layer models presented in this project can be adapted to apply to similar applications or use-cases, e.g. changing the human behaviour model to fit a virtual reality tennis match. Likewise, the structure of our modelling and simulator can be used to simulate entirely different applications. 

Obtaining such insights is a pre-requisite to set up guidelines for local network equipment deployment in order to support services in a cost-efficient manner. Moreover, the proposed framework constitutes a solid base for testing and development of new dynamic resource management methods, potentially \acs{AI}/\acs{ML}-based, and radio access parameter tuning strategies.

This work is a step towards autonomously managed network slices that independently configure the network and make trade-off-aware decisions to provide the best possible service with the available resources given the current network state. With ever-growing requirements, autonomous, flexible and optimal use of the network resources is a must to guarantee services, especially in the wireless access.


