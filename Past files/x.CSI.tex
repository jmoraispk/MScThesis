\section{Channel State Information}
\label{sec:CSI}


\todo[inline]{Contrast between access beams and fine tuned beams and assumption that we are connected and only managing beam alignment}


\todo[inline]{mention the usage of TDD, because the mmWave bands only have TDD and because allows UL-DL reciprocity. The concept that UL beams can be determined based on DL measurement, and vice-versa, is called beam correspondence [ https://ieeexplore.ieee.org/stamp/stamp.jsp?arnumber=8947954] and results in overhead reduction. Since beamforming and Massive MIMO have gained popularity in the community (cite emil stuffs), TDD has become increasingly favoured because it allows such reduction in beam management procedures. Thus, also here TDD operation is favoured.}


To form adequately oriented beams-pairs the most recent and precise channel state information is needed. There are two main ways of obtaining precise channel state information in 5G: the UL of a SRS, or the DL transmission of a CSI-RS. Naturally, since the approaches are based on different reference signals they involve different procedures, effectively resulting in different beams.


%%%%%%% SRS vs CSI %%%%%%%%%
% mention periodicities, pilot contamination problem with the SRS, ...

% critic MRT or ZF to a given antenna when in 5G

Acquiring CSI via the uplink of a \ac{SRS} allows the BS to have information on up to four distinct channels to the UE, because up to four orthogonal SRS sequences are available per UE \cite{DAHLMAN2018}. Therefore, the UE may use only up to four antenna ports, each antenna port mapping an SRS sequence to a set of physical antennas. In essence, this means that if the UE has four or less antennas, it can send a different SRS per physical antenna and the BS may extrapolate the channel to every single antenna of the UE. However, when the number of antennas on the UE is greater than four, it will not be possible for the BS to know the transformation that occurred between each UE antenna and each of its antennas. Many authors \cite{7504159} \todo{Add many more authors. Almost everyone does this. Including dear fellows Sjors and Remco. BE CAREFUL THAT SOME AUTHORS CONSIDER MORE THAN ONE SRS BUT THEY NEVER TALK ABOUT 5G!!! Only the authors that match zero-forcing/MRT with 5G and use more than 4 antennas are eligible for citation.} have the misconception that an infinite amount of orthogonal SRSs are available to be sent by the UE, and that is possible to obtain the channel to each antenna on the UE. However, when there is a limited amount of orthogonal sequences allowed per UE is the spatial filter applied to map one SRS to several physical antennas will be part of the transformation on the SRS, and the BS will only be capable of inverting the complete transformation that occurred on the reference signal. 

When UEs only had one or two antennas, this was no issue. But with the increase in popularity of mmWave frequencies, UEs may have hundreds of antenna elements. Therefore, the gain in CSI precision using SRSs will be marginal or even inferior to the gain using CSI-RSs, since CSI-RSs transmitted per UE may have 32 different ports, allowing for far more resolution. \todo{check nokia slides to be sure...}

But, in 
\todo{check these 2 papers: https://ieeexplore.ieee.org/document/7803878 and https://ieeexplore.ieee.org/document/8250021}

% IMPORTANT: mention partial vs full channel knowledge: one is having the 'best beam information'/having information fed-back. Other has complete information of the channel from reference signals coming from each of the UEs antennas. Then there CSI at the transmitter and at the receiver, but there is no need to speak about that because we will d

% talk also about the impulse responses (i.e. of digital filters) to explain the channel coefficients!!!! 



This comes with advantages, such as the possibility of performing a transmission optimised for that UE by targeting the antenna that has the highest potential for coherent constructive interference. But also disadvantages, e.g. considerable feedback overhead. This drawback is significant and thus not desirable in practice.



The CSI-RS-based feedback option is more flexible in terms of what information can be derived from that reference signal. Actually, even when acquiring CSI from SRSs, CSI-RSs are still used to derive other CSI like interference levels. The overhead burden of this option depends on the periodicity and the \acp{RB} assigned to CSI-RS measurements. The most relevant contrast with the previous option is that different CSI-RSs are transmitted in different beams, instead of each antenna. Therefore, in order to derive suitable beam-pairs the UE reports the beam that contains the strongest CSI-RS (and may even feedback the strengths of the next strongest beams). Although the beams created by the BS to transmit the CSI-RSs don't need to be pre-defined, i.e. based on a pre-existent GoB, given the feedback possibilities, pre-defined beams is the most likely approach.

% Talk about codebooks, non-codebook and so on....

In \cite{3gpp-codebooks} 3GPP standardises a GoB based on the two codebooks. The codebook of Type I supports until 8 layers per UE but has lower resolution/flexibility, making it a good fit for SU-MIMO operation. Codebook Type II fits MU-MIMO better, supporting 2 layers per UE and allowing further beam-steering precision. 
Nevertheless, in Section \ref{sec:GoB} we propose a simpler and more manageable grid as a first approach. The need for a more elaborate implementations should be carefully assessed since the scenarios, where our SXR meetings happen (indoors), and where the codebook-based precoders were meant to operate (outdoors), are significantly different.



% \cite{8947954}


% get that last email on the ports of srs and how multiple csi-rs are sent: each resource uses 32 codes (32 ports) and after that different resource elements
