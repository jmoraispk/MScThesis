\section{General View}
\label{sec:general_view}

This section concerns modelling and implementation aspects. Here's described in detail what decisions and assumptions were made along the way and how the simulator works.

Figure \ref{fig:diagram} illustrates the outline of the complete model. There are 3 main modules, the Radio Channel Generation Process in orange on the left, the Application Traffic model in blue on the right, and the 5G System Level Simulator at the bottom represented in green. Each of these parts will have its own section in this chapter.

% Insert big diagram.
\image{Implementation/Overview/overview.png}{placeholder for complete diagram}{fig:diagram}{.2}


In order, the 3 modules execute the following functions:
\begin{itemize}
    \item Radio Channel Generator - Configures user and antenna placement, models head position and orientation - important since the antennas will be in the head-mounted displays (HDM) -, aggregates all information in Quadriga tracks, creates Quadriga builders and generates the channel coefficients in time and frequency domain. Additionally, a parallelisation engine built in Python executes many compiled instances of the channel generator such that one can distribute workloads to perform more complex simulations quicker;
    \item Application Traffic Model - Emulates, in a general manner, the traffic that a Social-XR meeting would have. It's modelled as a video conference with more demanding latency and throughput requirements. The output of the model is packets arriving at time instants, which depends on application layer parameters such as compression, \ac{GoP} size, I to P frame size ratio, etc\dots
    \item System Level Simulator - Takes in the channel traces and the application requirements in terms of packets that require serving within a certain latency budget. Then updates buffers with the incoming packets, updates latencies of the queue, discarding packets that haven't been served in time, estimates the precoder for the this scenario, estimates the \acs{MCS} that should be used, the achievable SINR, the achievable bitrate. Finally, it computes what actually happened with the transmission, i.e. there were errors in some blocks, adjusts parameters to make estimations closer to the realisations, and updates queues and saves stats on a UE basis, TTI basis, and packet basis.
\end{itemize}