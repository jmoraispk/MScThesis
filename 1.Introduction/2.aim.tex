\section{Aim of this work}
\label{sec:aim}

We aim to study what are the best radio access configurations to optimally serve social \ac{XR} conferences. Here `configurations' consist of all parameters that control how radio resources are shared, ranging from algorithms to simple constants. Of course, such configurations are very scenario-specific, and the best match to one application, channel state and network state seldom is the best match to another.

\vspace{0.5cm}

To achieve that we need to derive the impact of each configuration on the \ac{QoS} for a given application. Logically, if it is known how each configuration impacts performance, it becomes trivial to choose the configurations that lead to the best performance. This is useful not only to optimise the physical deployment by saving costs, but also the management of that service, since ultimately the more efficient the provision of a service is, the less resources it requires to provide that service.

\vspace{0.5cm}

A solid way of obtaining insights about how such configurations impact performance is to model and simulate the application, the network equipment, the radio resource sharing mechanisms and the radio channel and then to measure the impact those configurations have on performance. This work aims to complete the first part consisting of modelling. The second part, which is based on extensive simulations, should be completed outside of this thesis.

\vspace{0.5cm}

More concretely, we introduce, implement and test a modelling framework for radio-layer optimisation and performance assessment in indoor social XR conferences. Such framework fills modelling gaps in the literature. We intend this work to be a stepping stone for future research on cellular communications, namely by providing a simulation environment not only for sensitivity analysis but also for development and testing of new management mechanisms, possibly AI-based.

\vspace{0.5cm}

Finally, although we are considering one specific application with a well-defined use-case, which we will clearly define and model the necessary components further ahead, we expect many of the conclusions to also apply to other applications. Furthermore, the methodologies here presented can be replicated to derive application-specific conclusions for different applications.





