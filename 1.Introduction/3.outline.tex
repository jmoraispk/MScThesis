\section{Outline}
\label{sec:outline}

This thesis is organised as follows.

In Chapter 1 we have motivated the relevance of studying radio layer optimisations to improve the performance of an \ac{XR} conference. Then we stated the plan to achieve it consists of extensively modelling all the relevant application, network and propagation aspects in order to enable ample impact analysis.

%How are we going to achieve the aim of this work and how to integrate it in the state of the art?
Chapter 2 reviews the state-of-the-art that allows us to achieve the objective of this work. First we survey XR applications requirements, architecture, user capture system, traffic and human movement models. Other virtual reality optimisation attempts are also compared with the aim of this work. Then we review the current best techniques and most promising technologies to achieve the demanding requirements. We start with a fundamental approach to enhancing throughput and we look at millimetre waves and massive MIMO as enablers of higher throughput and lower latency. Additionally we introduce formulations for channel matrices and beamforming.

Still in Chapter 2 we ... 

%We take the efforts to achieve it.
Chapter 3 is the methodology/modelling chapter.


%We present the results of our efforts.
Chapter 4 concerns current results. Even in its infancy this study already allows for some conclusions.


Chapter 5 we conclude and reiterate the most important results. Furthermore we suggest directions for future work.