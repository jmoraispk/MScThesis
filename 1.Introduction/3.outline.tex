\section{Outline}
\label{sec:outline}

In Chapter 1 we have motivated the relevance of studying radio layer optimisations to improve the performance of an \ac{XR} conference. We also integrated this study in a broader context by mentioning its applicability in management of future services in a virtualised and automatic manner. 

%How are we going to achieve the aim of this work and how to integrate it in the state of the art?
Chapter 2 provides a solid background for the main contribution of this research, contained in Chapter 3. Firstly, in Section \ref{sec:sxr_applications} we survey XR applications' requirements, packet traffic characteristic, aspects that influence the radio channel, namely human behaviour, and we look into optimisation attempts to VR applications performance from the physical layer perspective.

Subsequently, Section \ref{sec:key_tech} we review current radio layer techniques and most promising technologies to achieve the demanding requirements. After, Section \ref{sec:5gphy} presents how these techniques play a role in reality by examination of the relevant 5G physical layer standards and introducing radio access equipment, e.g. antenna systems. In Section \ref{sec:radio_channel} we survey radio channel simulators and find one that fits all our requirements. Lastly, the contributions we make to the state-of-the-art are listed.

%%%%%%%%%%%%%%%%%%%%%%%%%%%%%%%%%
% We take the efforts to achieve it.
Chapter 3 presents the Methodology. Here we disclose all modelling steps and assumptions. First we model the XR conference use case. We do so in Section \ref{sec:sxr_meeting_modelling} by addressing room sizes and how users are seated. Then model the antennas, user behaviour, and traffic. Section \ref{sec:propagation_environment} we use the selected channel generator to assess how the propagation environment changes in light of application use case assumptions, such as user position and behaviour.

Next, in Section \ref{sec:access} we present all functions executed by the network equipment to enable data transmission. We start off by stating how channel state information is acquired, how to create a grid of beams and select the best beam. Then user scheduling is addressed, consisting of how channel quality and instantaneous throughput estimation is done. Finally, we present a flexible and general framework to assess the quality of the transmission, we compute errors and we save the relevant metrics to assure good decisions also in the next transmission interval.

%%%%%%%%%%%%%%%%%%%%%%%%%%%%%%%%%%%%%%
% We present the results of our efforts.
In Chapter 4 we present results of an initial simulation study. To start, we clearly define the simulation in Section \ref{sec:scenarios} in view of the parameters introduced in Chapter 3. Then we investigate and compare a single and multiple user scenario, respectively, in Sections \ref{sec:single-user} and \ref{sec:multi-user}. We also discuss the results and take conclusions throughout.

Chapter 5 we conclude, reiterate the most important results and suggest directions for future work.