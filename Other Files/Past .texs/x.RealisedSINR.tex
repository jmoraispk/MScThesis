\section{Realised SINR}
\label{sec:realisedsinr}

To equate the Signal to Interference plus Noise power Ratio (\acs{SINR}), one must consider multiple users, \acp{BS}. Furthermore, it's important to account for impact of precoders and the flexibility of multi-layer transmission, which are left out in most \acs{SINR} derivations. In this section is a formulation for the downlink with those in mind.

The following notation is with respect to a single \acs{PRB}, hence we can drop any time or frequency notation.
As a clarification, codewords are transport blocks that undergo a coding procedure. Afterwards, each codeword is scrambled and each set of 2, 4, 6 or 8 bits, depending on the modulation, is mapped to a constellation symbol. To make up an \acs{OFDM} symbol, 12 of symbols are aggregated in frequency, one for each (orthogonal) subcarrier. Then 14 \acs{OFDM} symbols are aggregated in time to constitute a \acl{PRB}. These portions of resources in time and frequency are attributed to different layers. Codewords are distributed across layers, which we can regard as independent data streams that use certain \acsp{PRB}. Therefore, it is only possible to transmit several layers to a user if there are paths sufficiently independent from the \acs{BS} to that user, or else there'll be interference at the reception because layers relying on separation on the spatial domain to use the same time and frequency resources.

Bold-face is used to represent matrices or vectors, respectively, for uppercase and lowercase letters. As such, we denote:
\begin{itemize}
    \item Cell assignment - $\mathcal{M}_b$ as the set of devices co-scheduled by the \ac{BS} $b \in \mathcal{B}$, with $\mathcal{B} = \{1,..., B\}$;
    \item Transmit powers - $P_{bmv}^{tx}$ as the transmit power used by \ac{BS} $b$ to mobile device $m$ in layer $v$;
    \item Transmit Precoding - $\bm{W}_{bm}$ as the $I_b \times V_{bm}$ precoding matrix, used by \ac{BS} $b$ to device $m$, where $I_b$ is the number of antennas in \ac{BS} $b$, and $V_{bm}$ is the number of layers scheduled from $b$ to $m$. The $v$-th column vector $\bm{w}_{bmv}$ is the precoder used in layer $v$.
    \item Receive Precoding - $\bm{B}_{bm}$ as the $V_{bm} \times J_m$ combining/receive precoding matrix used by $m$ to receive from BS $b$, having $J_m$ as the number of antennas in the mobile terminal $m$. Now the $v$-th row, $\bm{b}_{bmv}$, is the precoder used to receive layer $v$.
    \item Channel - $\bm{H}_bm$ is the complex-valued channel response matrix that describes the amplitude and phase changes (by means of a complex value) that occur between each of the $I_b$ antennas in \ac{BS} $b$ and $J_m$ antennas in \ac{UE} $m$. Therefore, $\bm{h}_{bmij}$ is the entrance $(i,j)$ of the matrix, $\bm{h}_{bmi}$ is the $1 \times J_m$ row vector that can be used to precode from $m$ to the $i$-th antenna of $b$, and $\bm{h}_{bmj}$ is the $I_b \times 1$ column vector that associates $b$ to the $j$-th antenna of $m$.
\end{itemize}


We'll proceed to compute the received power, the intra and inter-cell interference and the noise.

The received power depends on the receive beamforming, the transmit precoding, the channel and the transmit power used in layer $v$ between \ac{BS} $b$ and device $m$. It's given by \eqref{rx_pow}.

\begin{equation} \label{rx_pow}
    P_{bmv}^{rx} = \left|\bm{b}_{bmv} \left( \bm{H}_{bm} \bm{w}_{bmv} \right)\right|^2 P^{tx}_{bmv}
\end{equation}


We concretise transmit and receive precoding in the subsequent section (\ref{sec:precoding}), we elaborate on how to determine $\bm{b}_{bmv}$ and $\bm{w}_{bmv}$ and their matrices. 

Furthermore, the quantity between vertical brackets in \eqref{rx_pow}, represents the transformation the transmitted symbol suffers from the time it is transmitted until its detection (see section \ref{sec:comp_deduction} for a precise derivation). Computing the power of the detected symbol leads to the square of the absolute value of that transformation multiplied with the transmit power of that symbol, which is the structure we have in \eqref{rx_pow}. For convenience, for the Intra and Inter-cell interferences, only the transformation is computed. Then, in the complete \acs{SINR} expression, the square of the absolute value of the transformation and the transmit powers are taken into account.

The intra-cell interference $\bm{C}_{bmm'}(v,v')$ is caused by $b$ in user $m$ in layer $v$ from the signals transmitted by the same \acs{BS} to other users $m'$ in their user-specific layers $v'$. The complex matrix $\bm{C}_{bmm'}$ is $V_{bm} \times V_{bm'}$ and in each entrance has the cross-layer interference of any 2 users. Note, if $m = m'$, then $\bm{C}_{bmm'}(v,v')$ would be represents the interference caused by the user's own layers. Along the diagonal, where $v = v'$, the entrances are unitary because the interference is maximal. If perfect layer isolation is achieved, all other entrances are null. Intra-cell interference be expressed as \eqref{intra_interference}.

\begin{equation} \label{intra_interference}
    \bm{C}_{bmm'} = \bm{B}_{bm} \left( \bm{H}_{bm} \bm{W}_{bm'} \right) 
\end{equation}

The inter-cell interference $\bm{D}_{bmb'm'}(v,v')$ caused in the transmission from $b$ to $m$ in layer $v$ comes from transmissions from other cells $b'$ to all \acsp{UE} they are serving $m' \in \mathcal{M}_{b'}$ in layers $v'$, with $v' \in \left\{1, ..., V_{b'm'} \right\}$. It's given by \eqref{inter_interference}. 

\begin{equation} \label{inter_interference}
    \bm{D}_{bmb'm'} = \bm{B}_{bm} \left( \bm{H}_{b'm} \bm{W}_{b'm'} \right)
\end{equation}


The noise power is scaled by the received precoder used. We assume white noise, computed by Johnson-Nyquist formula \cite{PhysRev.32.110} $k_B T B$. Representing the noise variance by $N$, the complete SINR expression for the reception of layer $v$ in device $m$ by \acs{BS} $b$ is given in \eqref{sinr_single_layer}

\begin{equation} \label{sinr_single_layer}
\begin{split}
    &\text{SINR}_{bmv} = \\
    &\frac{\left|\bm{b}_{bmv} \left( \bm{H}_{bm} \bm{w}_{bmv} \right)\right|^2 P^{tx}_{bmv}}{\mathlarger{\sum}_{\ m' \in \hspace{.05cm} \mathcal{M}_b} \hspace{-0.2cm} \sum_{v' = 1}^{V_{bm'}} \left| \bm{C}_{bmm'}(v,v')\right|^2 P^{tx}_{bm'v'} + \mathlarger{\mathlarger{\sum}}_{\substack{b' \in \hspace{.05cm} \mathcal{B} \\ b' \neq b}} \hspace{-0.1cm} \mathlarger{\sum}_{\ \hspace{.05cm} m' \in \hspace{.05cm} \mathcal{M}_{b'}} \hspace{-0.35cm} \sum_{v' = 1}^{V_{b'm'}} \left| \bm{D}_{bmb'm'}(v,v') \right|^2 P^{tx}_{b'm'v'} + N \left| \bm{b}_{bmv} \right|^2 }
\end{split}
\end{equation}

And, for implementation sake, the vectorised variant to obtain directly the SINR in all layers of the receiver $m$ is in \ref{sinr_multi_layer}. Naturally, adaptation to matrices is required. Thus, $|\cdot|_{\text{e.w}}^{\odot2}$ represents Hadamard exponentiation of the absolute values, which is no more than taking the absolute value of each element and squaring it. Also, $|\cdot|_{=}^2$ signifies the row-wise norm, the norm of each row $\bm{b}_{bmv}$ appended in a column vector.

% if the summation sign is small, the limits can go inline. use \limits to prevent that
\begin{equation} \label{sinr_multi_layer}
    \text{SINR}_{bm} = \frac{\left|\bm{B}_{bm} \left( \bm{H}_{bm} \bm{W}_{bm} \right)\right|_{\text{e.w}}^{\odot2} \bm{p}^{tx}_{bm}}{\mathlarger{\sum}_{\ m' \in \hspace{.05cm} \mathcal{M}_b} \left| \bm{C}_{bmm'}\right|_{\text{e.w}}^{\odot2} \bm{p}^{tx}_{bm'} \hspace{.05cm} + \hspace{.05cm}  \mathlarger{\sum}_{\substack{b' \in \hspace{.05cm} \mathcal{B} \\ b' \neq b}} \sum\limits_{\ \hspace{.05cm} m' \in \hspace{.05cm} \mathcal{M}_{b'}} \left| \bm{D}_{bmb'm'} \right|_{\text{e.w}}^{\odot2} \bm{p}^{tx}_{b'm'} \hspace{.05cm} + \hspace{.05cm} N \left| \bm{B}_{bm} \right|_{=}^2 }
\end{equation}

In the above expression, note that it's used column vectors of transmit powers. Also, the fraction actually symbolises an element-wise division, since both numerator and denominator are column vectors.








\subsection{Precoding} \label{sec:precoding}
MRT, ZF, some MMSE?
MRC could be used at the reception, but inverting maybe it's more realistic. 
In any case, a pseudo-inverse is computed.


%The receive beamformer $\bm{b}_{bmv}$ was optimal for the reception of data from a couple of TTIs ago, therefore that should cause a slight degradation in the SINR.

See my notes in the big sheets of paper, complete matrix deduction there.



\subsection{Complete Deduction} \label{sec:comp_deduction}
From symbols to powers. See Haibin doc and notes on 'why $\sum|w_i|^2 = 1$ guarantees no power scaling'.