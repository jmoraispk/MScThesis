\newpage
\section{System-level Simulator}
\label{sec:sls}
\todo{Rename according to new structure}

In this section, we detail the functions executed by the network equipment during a simulation. The network equipment  receives the radio channel coefficients across time from the channel generator and the application requirements in a form of packets to arrive in certain \acp{TTI}. The BS has to handle important tasks such as the update of the Channel State Information and the numerous procedures related with individually processing the transmission of each \ac{TTI}, in order to imitate the radio connection present in a XR Conference. 




See Figure \ref{fig:flowchart_sls} for a flowchart with an overview of all the steps required to simulate a DL transmission. These steps are detailed in this Section.

\image{SLS/flowchart.png}{Flowchart for of simulation steps for each TTI.}{fig:flowchart_sls}{.6}

We've modelled both DL and UL transmissions, and both are the same in most steps. In every step where UL and DL procedures differ, that difference is pointed out. Thus, if nothing is said about downlink or uplink, assume the process is equal in both cases.

\todo[inline]{Move part of the next few paragraphs to the 5G section in the literature review + add what a frame, subframe and slot are.}

To start with, at the beginning of each TTI, the TTI is identified as a DL or UL TTI. This implies a hard TDD split; 'hard' in the sense that there are no flexible slots where UL or DL transmissions are allowed interchangeably, only slots with a pre-determined transmission direction. And the pre-determined transmission direction is given by position of the slot in the slot period $P_\text{slot}$ (how many slots in certain slot pattern) and the TDD (slot) split $s_\text{TDD}$ (proportion of DL and UL slots in a slot period). Different from LTE, 5G allows a slot-based periodicity. This means that a slot pattern may be repeated within a frame. A slot pattern, in it simplest formulation, is defined from a number of downlink slots $N^\text{UL}_\text{slots}$ followed by a number of uplink slots $N^\text{UL}_\text{slots}$ \cite{3gpp-slot_periodicity}. The ratio of the former over the latter is known as the TDD split, commonly represented as $N^\text{DL}_\text{slots} : N^\text{UL}_\text{slots}$ (e.g. 5:1). And their sum, is the length of the slot period $L_\text{slot}$. See Equations \eqref{eq:tdd_split} and \eqref{eq:slot_per}.

\begin{align}
    s_\text{TDD} &= N^\text{DL}_\text{slots} \ / \ N^\text{UL}_\text{slots} \label{eq:tdd_split} \\
    P_\text{slot} &= N^\text{DL}_\text{slots} + N^\text{UL}_\text{slots} \label{eq:slot_per}
\end{align}

% This crap came from https://www.sharetechnote.com/html/5G/5G_tdd_UL_DL_configurationCommon.html

In \cite{3gpp-frames_and_slots} (Table 11.1.1-1) 3GPP defines the slot formats for normal cyclic prefix slots. From this table, Figure \ref{fig:slot_format} represents slot formats 0, 1, 28 and 34, respectively. 


\image{SLS/slot_format2.png}{slot format}{fig:slot_format}{.42}

Other formats consist in changing the number of DL, UL or Flexible (F) symbols, keeping the order of having DL slots first, then flexible slots and then UL slots. All slot types are optional.

We use the top two formats represented in Figure \ref{fig:slot_format} and we assume the HARQ feedback for retransmission is given in the same TTI, such that if TTI $t$ had errors, the same information can be transmitted in tti $t+1$. \todo{introduce HARQ delays in 5G section}

\todo{
    Another approach would be to choose the bottom format and had a bit rate correction by fraction of 1/14 or 2/14, respectively, 7\% or 14\%. 
}


However, one must not forget to ask what is a good TDD split $s_\text{TDD}$ and what is a length for slot period, $L_\text{slot}$. The TDD split should match the throughput requirements, i.e. if the aggregated information to send in the downlink totals twice the information to send in the uplink, then a 2:1 TDD split should offer the best results, from this \textit{à priori} analysis. Regarding the frame size, mind the structure defined by 3GPP, where first all DL slots are . On one hand, the frame size can't be too extensive, or some UEs may starve of throughput and expire the packet latency budgets, leading to excessive packet drops. On the other hand, the frame size must be big enough to support the TDD split defined above. Given the application specifics defined the \ref{sec:at} referring to how the total UL load is proportional to the number of physically present users and DL load changes proportionally to the number of participants, including remote participants too, the TDD split should be chosen in conformity with the scenario specifics. 


Before processing the actual TTI it is verified if CSI should be updated, which happens every $L_\text{CSI}$ TTIs. Then, it is also checked if the current scheduling information for that TTI is to be updated. Only after those two procedures is the present TTI processed. Firstly, how CSI is updated and how the scheduling decisions are made.

\subsection{Update CSI}
Only in the event that the previous \ac{TTI} is a \ac{CSI}-collection \ac{TTI}, there is an update on the best beam pairs between the UE and the BS(s), determining the best directions for transmission, and refreshing the interference measurements. 

To update the best beam pairs between UE and BS, several details must be laid down. Firstly, we consider that each BS transmits each polarisation separately. This assumption derives from the fact that at least two independent streams/layers must be supported per UE, and the only way of not sacrificing performance while guaranteeing orthogonality in polarisation domain. Hence, the BS sends CSI-RSs coded into $N_\text{CSI Beams}$ beams, per polarisation. 3GPP defines up to four of the beams received the best can be fed back to the BS. We are stepping into manufacturer's implementation domain perhaps, but we assume at most four per polarisation.

Thus, in essence, each BS sends schedules CSI resources on each UE and sends at most four CSI-RS per polarisation, coded in different beams. However, to facilitate beam choice, we make decisions based only on the best beam reported per BS , per polarisation (i.e. $N_\text{CSI Beams} = 1$). See Section \ref{sec:future-work} on other possible choices that can lead to increased performance.

The best beam-pair, per BS polarisation is chosen by maximising the channel gain achieved from performing a transmission with a given GoB beam, over the channel of that polarisation, and doing best effort at the reception to received that signal, using Maximum Ratio Combining (MRC). MRC, as shown in Equation \eqref{eq:mrc}, leads to a purely real signal with amplitude equal to the square of the norm of the incoming signal. This means that because we are computing the UE-side beamformer $\bm{w}^\text{UE}$ as an MRC taking into account the transmit-side beamformer, to maximise the norm of the received signal it is sufficient to choose the appropriately the transmit-side beamformer. Additionally, provided that all beamformers, by definition, are normalised (making the total radiated/received power not changing), the internal product of a beamformer $\bm{w}$ that belongs to the set of beamformers in BS $b$ \ac{GoB} $\mathcal{W}^\text{GoB}_b$ (defined in Section \ref{sec:GoB_and_csi}) with the channel coefficients $\bm{h}_{b m l}$ leads to the norm of the signal arriving to each of the UEs antennas. Equation \eqref{eq:GoB_choice} summarises this logic and the choice of precoder for the BS for a case where $N_\text{CSI Beams} = 1$, and $\bm{w}^\text{UE}$ is a MR beamformer.

\begin{equation} \label{eq:GoB_choice}
    \bm{w}^\text{BS}_{bml} = \argmax_{\bm{w} \ \in \ \mathcal{W}^\text{GoB}_b} \left(\left| \bm{w}^\text{T} \cdot \bm{h}_{b m l} \cdot  \bm{w}^\text{UE}_{bml} \right|\right) = \argmax_{\bm{w} \ \in \ \mathcal{W}^\text{GoB}_b} \left( \bm{w}^\text{T} \cdot \bm{h}_{b m l} \right)
\end{equation}

An analogous procedure is done to discover the $N_\text{CSI Beams}$ best beams.

The beamformer on the BS side is always a beam steering vector from the GoB, and the UE-side precoder is always the Maximum Ratio beamformer that fits perfectly the use of the BS beamformer and the channel. The UE precoder is given by the MRT/MRC (respectively, in UL/DL) correspondent computation, in Equation \ref{eq:GoB_choice_ue}.


\begin{equation} \label{eq:GoB_choice_ue}
    \bm{w}^\text{UE}_{bml} = \frac{\left(\bm{h}_{b m l} \cdot \bm{w}^\text{BS}_{bml} \right)^H}{\left|\bm{h}_{b m l} \cdot \bm{w}^\text{BS}_{bml} \right|}
\end{equation}


The beam-pair computed in this way is used for UL and DL transmissions exploiting TDD reciprocity.

The result from the update is a list of the best $N_\text{CSI Beams}$ between each UE and each BS, per BS polarisation.


It is relevant to talk about how multi-layer transmission is modelled. In the DL, not all antenna elements are used for each transmission. Only half of the elements transmits in each polarisation, the half that has the correspondent orientation. Provided that the power radiated per layer is not changed by different numbers of elements for the transmission, then to consider half the antenna elements it is not problematic. In the UL, the BS can receive using both polarisations for the reception. In doing so, it combines element contributions by using the same precoder in each polarisation and adding up the signals. 

Furthermore, in this work we consider a maximum of two independent layers per UE, and that those layers can must be sent by the same BS. Although that is a limits the flexibility and the purpose of multi-BS simulations, it has to be left as future work (see Section \ref{sec:future-work}).


Regarding the interference measurements update, it is used the interference from $\tau_\text{TTI}$ TTIs ago. A major disadvantage of estimating the interference in this manner comes from the fact that the experienced interference is extremely on which UEs are scheduled and what beams are used in that given TTI. If the scheduled UEs or beamformers in use change, then is expected a major change in the experienced interference to take place. We foresee precise interference estimation algorithms, perhaps driven by learning mechanisms, to be a future direction of work. We discuss this matter further in Section \ref{sec:future-work}.

Despite this last inevitable drawback, the modelling of \ac{CSI} acquisition and feedback is coherent and realistic.





\subsection{Update Scheduling}
Analogous to the 'Update CSI' procedure, the 'Update Scheduling` phase is only executed in the respective TTIs, depending on the scheduling periodicity. Some authors consider updating the scheduling information every two TTIs \cite{7986957}, to match LTE. Other authors \cite{dmimo-remco} prefer updating as frequently as possible. We define a parameter associated with the scheduling periodicity, the length of scheduling period $L_{sch}$ (in TTIs), to be tuned.

The scheduling procedure is described by the following steps:

\begin{enumerate}
    \item List UEs to consider for scheduling - only UEs with non-empty buffers are examined to be part of the scheduled UEs list. 
    \item Select best BS(s) per UE - the BS with the best beam-pair for a given polarisation in the UE is considered as the serving BS for that polarisation.
    \item Select the number of layers per UE (SU-MIMO setting) - either a single layer or a dual-layer setting is selected. The setting that gives highest aggregated bitrate is chosen. If only one layer is transmitted, it can be transmitted with twice as much power per antenna element as the power used in dual-layer transmission. The selected polarisation in the BS for a single-layer transmission is the one that has the highest SINR to the UE, while in the dual-layer transmission there will be a bitrate per data stream. Therefore, the single-layer stream will have an higher bitrate that each of the streams in a dual-layer mode, but the aggregation of the two stream in dual-layer transmission may total more bits across.
    
    The process to estimate the achievable bitrate from a SINR is used later once more, and is done taking into account the number of PRBs scheduled UE. We use wideband scheduling, attributing all resources to every transmission, hence the number of PRBs in the available band is used for scaling the bitrate. 
    
    The procedure for bitrate estimation is the following:
    \begin{enumerate}
        \item Use the SINR to estimate the MCS from the MCS curves - the first \ac{CQI} that achieves a lower percentage of block errors than the \ac{BLER} target $BLER_0$ is chosen. Represented in Figure \ref{fig:blercurves} are the BLER curves for all CQIs considered. The point at which each MCS curve intercepts the BLER probability of 10\% is marked. The MCS that corresponds to each CQI index is present in Table \ref{tab:mcs}.
    \end{enumerate}
\end{enumerate}


\image{SLS/MCS_final_res300.png}{BLER curves for all MCSs. Simulated with Vienna Link-Level Simulator \cite{Vienna5GLLS}.}{fig:blercurves}{.37}

% THIS IS HOW TO PUT IMAGES IN BETWEEN LISTS!
% use this nice itemize property to continue the numbers properly
\begin{comment}
\begin{itemize}
    \item[1] de
    \item[(a)]  df
    \item[4.] 
\end{itemize}
\end{comment}

\begin{itemize}
    \item[] \begin{itemize} %\setcounter{enumii}{1} % note the depth 2 counter!
        \item[(b)] Adjusts the choice of \ac{MCS} with the UE-specific \ac{OLLA} parameter - every time a MCS is estimated, it is adjusted with the OLLA parameter at the last step. The OLLA parameter for a UE is initialised at zero and is updated in each TTI the given UE is scheduled (see Equations \eqref{eq:ola_update1_success} and \eqref{eq:ola_update1_fail} at the end of the section). The parameter adjusts the MCS choice by flipping an appropriately biased coin and adding either $\floor{\Delta_{OLLA}}$ or $\ceil{\Delta_{OLLA}}$ to the CQI index estimated in the previous step. An appropriately biased coin in this situation is a coin that selects to round down the OLLA parameter with a probability of $\ceil{\Delta_{OLLA}} - \Delta_{OLLA}$. This makes sense because $\Delta_{OLLA}$ is decreased when a block has errors, thus making more likely that the MCS is reduced when the link has worse quality than expected. When the block doesn't have errors, it makes it more likely to increase the MCS estimate, such that a good link condition can be taken advantage of to increase the bitrate. Note that this formulation still works as supposed for negative values.

        \item[(c)] From the newly adjusted MCS and the assigned resources, estimate the achievable bitrate - 
        We need a set of MCSs from which to choose MCSs. In \cite{3gpp-codebooks} there are several tables with different sets MCSs, and we've picked Table 5.2.2.1-3 because it is the only one that has 256QAM modulation, the highest modulation order in 5G. This is something desirable because the scenario should allow very high SINRs and we don't want to limit the maximum bitrate by using a lower MCS than it would be possible to use. Table \ref{tab:mcs} shows the previously mentioned MCS table and includes the ideally achieved throughputs, when all resource elements are used for data transmission. The last column of the table is obtained for one PRB-worth of \acsp{RE} (12 subcarriers used for the duration of 14 symbols results in 168 time-frequency resource elements, or transmitted symbols). Thus, we consider 168 symbols for data in each PRB ($N_\text{symb}^\text{PRB} = 168$). Since each symbol codes $\log_2(M)$ bits, where $M$ is the modulation order, we get the number of information bits transmitted per symbol $N_\text{info bits}^\text{symb}$ from multiplying the number of bits coded in each symbol $N_\text{bits}^\text{symb}$ to the code rate $R_\text{c}$ (percentage symbols used for information) - results in the efficiency column. From the product of the efficiency with $N_\text{symb}^\text{PRB}$, we obtain the bits per slot $N_\text{bits}^\text{slot}$, which is independent of the numerology (i.e. the number of bits sent in each slot is independent of the slot duration). Dividing by the slot duration $T_\text{slot}$ we obtain the bitrate $R_\text{B}$. Equation $\eqref{eq:bitrate}$ sums up this process.


% KEEP THIS SENTENCE In \cite{3gpp-codebooks} there are several tables with different sets MCSs, and we've picked Table 5.2.2.1-3 because it is the only one that has 256QAM modulation, the highest modulation order in 5G. This is something desirable because the scenario should allow very high SINRs and we don't want to limit the maximum bitrate by using a lower MCS than it would be possible to use. 


        \begin{equation} \label{eq:bitrate}
            R_\text{B} = \frac{\log_2(M) \times N_\text{bits}^\text{symb} \times R_\text{c} \times N_\text{symb}^\text{PRB}}{T_\text{slot}}
        \end{equation}

        % https://extracttable.com/ is awesome!
        \begin{table}[h]
            \centering
            \caption{Table 5.2.2.1-3 from \cite{3gpp-codebooks} with the maximum achievable bitrate in a 1ms slot.}
            \label{tab:mcs}
            \begin{tabular}{|c|c|c|c|c|} 
            \hline
            CQI index & Modulation & Code rate x 1024 & Efficiency & Bit Rate [kbps]  \\ 
            \hline
            0         & \multicolumn{4}{c|}{out of range}                             \\ 
            \hline
            1         & QPSK       & 78               & 0.1523     & 25.59            \\ 
            \hline
            2         & QPSK       & 193              & 0.377      & 63.33            \\ 
            \hline
            3         & QPSK       & 449              & 0.877      & 147.33           \\ 
            \hline
            4         & 16QAM      & 378              & 1.4766     & 248.06           \\ 
            \hline
            5         & 16QAM      & 490              & 1.9141     & 321.56           \\ 
            \hline
            6         & 16QAM      & 616              & 2.4063     & 404.25           \\ 
            \hline
            7         & 64QAM      & 466              & 2.7305     & 458.72           \\ 
            \hline
            8         & 64QAM      & 567              & 3.3223     & 558.14           \\ 
            \hline
            9         & 64QAM      & 666              & 3.9023     & 655.59           \\ 
            \hline
            10        & 64QAM      & 772              & 4.5234     & 759.94           \\ 
            \hline
            11        & 64QAM      & 873              & 5.1152     & 859.36           \\ 
            \hline
            12        & 256QAM     & 711              & 5.5547     & 933.19           \\ 
            \hline
            13        & 256QAM     & 797              & 6.2266     & 1046.06          \\ 
            \hline
            14        & 256QAM     & 885              & 6.9141     & 1161.56          \\ 
            \hline
            15        & 256QAM     & 948              & 7.4063     & 1244.25          \\
            \hline
            \end{tabular}
        \end{table}

    \end{itemize}
\end{itemize}
\begin{enumerate} \setcounter{enumi}{3}
    \item Compute UE priorities based on the estimated instantaneous bitrate - provided that our traffic is all real-time traffic from the application, and given that our application is very latency-constrained, \cite{7020879} advises the use of a latency aware scheduler and shows that \ac{M-LWDF} performs slightly better than \ac{EXP/PF}. Nonetheless, the classic \ac{PF} scheduler and the two mentioned before are good options. See Appendix \ref{ap:d} for their details.

    \item Select which users to co-schedule (MU-MIMO setting) - to list the users to be scheduled together until the next update to the schedule. The co-scheduling rule for a single-BS operation is to add one UE layer at a time to the list, by order of \ac{UE} priority (computed in the previous step), if the best beams used for those layers are compatible with the previously added \ac{UE} layers. And we define as compatible beams when the BS-side beam, belonging to the \ac{GoB} is at least than $\kappa$ beams apart, with $\kappa \in \mathbb{N}_0$. If $\kappa$ is 0, then the all layers are accepted. If $\kappa = 1$, then the beams must be different - adjacent beams have a distance of 1, so are still used together. Beams located diagonally adjacent of the \ac{GoB} are considered to have a distance of 2, hence they may be co-scheduled when $\kappa \leq 2$. Figure \ref{fig:beam-rule} illustrates the beams that can't be co-scheduled with certain values of $\kappa$, representing in filled circles as incompatible beams, and empty circles as compatible beams with the central orange beam. More generally, the beam distance is defined by the sum of absolute differences of the beam indices in the grid. Mathematically, the beamformers $\bm{w}_{i,j}$ and $\bm{w}'_{i',j'}$, having $(i,j)$ and $(i', j')$ as the GoB indices, respectively, are compatible if $|i-i'| + |j - j'| \geq \kappa$.

\todo[inline]{IMPORTANT!!!!!! There's quite a big problem here. How to know if both layers can be co-scheduled or not? This actually starts in the SU-MIMO setting: just having different polarisations does not guarantee enough orthogonality such that the interference is insignificant/bearable!}

\end{enumerate}
    
\image{SLS/mumimo-beam-rule.png}{adf}{fig:beam-rule}{.5}

\begin{enumerate} \setcounter{enumi}{6}
    \item Power control - Depending on the beamforming strategy, it may be necessary to scale down all precoders due to excessive power constraints. This, however, does not concern us because hour beamformers always have uniform amplitude. The power control process in the downlink is as simple as distributing the maximum total transmit power equally amongst the scheduled UEs, and equally amongst UE layers. In case of UL, the process is a tad more complex: each UE always transmits with full power and gets assigned more or less resources for the transmission depending on its the link quality. We apply an open loop power control scheme with fractional path loss compensation suggested in \cite{ul_power_control}. Equation \eqref{eq:ul_power_control} gives the transmitted power spectral density $S_t$ from the estimated path loss $PL$, the power compensation factor $\alpha_P$, and the target received power density $P_0$. A good combination of parameters for $P_0$ and $\alpha_P$ is, respectively, -80 dBm/PRB and 0.8.
    
    \begin{equation} \label{eq:ul_power_control}
        S_t = P_0 + \alpha_P \cdot PL_\text{[dB]}
    \end{equation}


    Then, given a constant transmit power per UE of $P_t^\text{UE}$ and an achievable power spectral density, we get the number of PRBs to request the BS $N_\text{req PRB}$, as shown in Equation \eqref{eq:UL_prbs}.

    \begin{equation} \label{eq:UL_prbs}
        N_\text{req PRB} = \frac{P_t^\text{UE}}{S_t} 
    \end{equation}

    Having all the scheduled UEs doing a similar request for resources, either the total requested number of PRBs is smaller than the total available, and each UE gets what it requested. Or the requests exceed the available and all requests are scaled down equally.

    \todo[inline]{this doesn't really make a lot of sense. The scheduled UEs are 'compatible' from a MU-MIMO point of view. Therefore, there would be no interference. The contention happens in the UEs that were NOT scheduled due to beam incompatibilities yet they can be due to this resource separation... It makes sense to only attribute the necessary PRBs, but there should always be PRBs for all UEs since they have, supposedly independent paths}

    \item Re-estimate the \acsp{SINR}, per \ac{UE}, per layer - the power to be used in each layer is now known, and the assigned spectrum per UE is also known (applicable in UL since DL performs wideband scheduling). Therefore, a more accurate SINR estimate is made.
    
    \item Choose the final MCS, per UE, per layer, based on the last SINR estimate - analogous to step three. 
\end{enumerate}


As a final part of the scheduling process, the estimated achievable bit rates are computed again and used to update scheduler parameters, e.g. the averaged bitrate for the fairness ratio. See Appendix \ref{ap:d} to know how the update is made. The scheduler that computes UE priorities should be updated every time at the end of the scheduling step such that it can perform accurate choices to maintain fairness between users, maximise the throughput and meet latency requirements. If sub-band/narrowband scheduling is done in the future, then the scheduler needs to be updated after scheduling each sub-band.



\begin{comment}

There are three noteworthy remarks about the above:
\begin{itemize}
    \item It can be performed on a narrowband level: we may do the above for portions of the band. The only thing that needs to change across the schedulable bands is the aggregated-across-time throughput metric in the scheduler, or else the resources would be attributed equality anyway. Therefore, instead of the actual aggregated-across-time throughput, we should use and estimated aggregated-across-time throughput, and update that estimation with the expected bitrate each user would get from the schedulable narrowband. For the first portion of band, the estimation and the actual values would be the same. Note that if the scheduler takes into account latencies as well, then the head-of-queue delay value provided to the scheduler must change across narrowbands, it should assume the bits are correctly sent and compute the latency of 'the next packet on the queue'. Or something even smarter, perhaps like considering a few more packets ahead of time, since it will take a while until the scheduling changes.
    
    
    \item It can be performed for UL as well, with small changes:
    \begin{itemize}
        \item The \acsp{PRB} attributed should depend on the channel quality of the UE, and a Power Density should be computed to achieve a certain SINR per \acs{PRB}. \todo{cite paper and input common values}
        \item The interference needs to be computed differently for the uplink. Same principle, but different variables since other UEs are causing the interference. Furthermore, the wideband interference estimates can be adjusted to equate the total interference in the UL band;
        \item The noise needs to be computed differently since we won't know how big the bandwidth will be ahead of time. For the first estimation, it is enough to consider the same bandwidth for all users, or to estimate based on the quality of the beam-pair (channel gain measured when CSI measure was obtained). Then, when the actual priorities are defined, a new and more precise value can be used for the noise. Moreover, since interference is expected to me far more relevant than noise, the wideband noise can be considered for both cases, giving a marginally different estimate, even though it is a worst case scenario estimate noise-wise.
        \item The transmit power in the UL is always the same, it may be more or less distributed across frequency. Unless the in SINR is limited by too much interference and using the less power results in similar SINRs, which is rarely the case, the transmit power can be always maximum.
    \end{itemize}
    
    
    \item It can be performed for Implicit beamforming, with small changes:
    \begin{itemize}
        \item The first SINR estimation is performed with \ac{MRT} precoders;
        \item The second SINR estimation is performed with \ac{ZF} precoders - that computation requires the MU-MIMO setting;
        \item Anything else?! 
    \end{itemize}

    \item When more BSs are used, and when more beams besides the best are reported, something more intelligent can be made in the BS selection. Nonetheless, it is assumed that all BSs communicate at a very fine time scale between them, to choose how to best serve the UEs.
\end{itemize}

\end{comment}


\subsection{Simulating the TTI}

At the end of the processing per TTI, it is time to obtain the outcomes of the realised transmissions: the SINRs each UE got in each layer in each PRB and how those resulted in the success or failure of the transport blocks. The steps follow:
\begin{enumerate}
    \item Obtain the \ac{TBS} - we get the number of bits sent in the slot in the numerator in Equation \eqref{eq:bitrate}, and the \ac{TBS} is a fraction of this number. Note that if we have only one \ac{TB} per slot, our bitrate will either be one from the table (if $T_\text{slot} = 1\text{ms}$), or zero, respectively, if the block has been successfully transmitted or not. Thus, we define the number of transport blocks per slot $N_\text{TB}^\text{slot}$ and it related with the size of each \ac{TB} $L_\text{TB}$ as shown in Equation \eqref{eq:TBS}.
    
    \begin{equation} \label{eq:TBS}
        L_\text{TB} = \frac{N_\text{bits}^\text{slot}}{N_\text{TB}^\text{slot}}
    \end{equation}

    This such manner, $N_\text{TB}^\text{slot}$ TBs are sent per slot and the experienced bit rates depend on how many of them are delivered with no errors. 
    
    \item Compute the realised SINR, for each scheduled \acs{UE} and for each assigned \ac{PRB} - we know how to compute the received power $P_r$ in a transmission between BS $b$, UE $m$, in layer $l$, provided the transmitted power $P_t$, BS beamformer $\bm{w}^\text{BS}$, UE beamformer $\bm{w}^\text{UE}$ and channel matrix $\bm{H}_{bml}$. Furthermore, notice this formulation is not UL or DL specific. And knowing how to calculate this quantity, we are capable of calculating the signal power $P_s$, the intra-cell interference, the inter-cell interference power $P_\text{InterCI}$. 
    Assuming we intend to compute the SINR in the $i$-th PRB scheduled between BS $b$ and UE $m$ in layer $l$, we would do the following on a PRB-basis:

    \begin{enumerate}
        \item Compute \ac{RSS} $P_s$ - power received on the link we want to compute the SINR. 
        \item Compute Inter-layer interference $P_\text{ILI}$ - in case of multi-layer transmission, other layers scheduled to/from the same UE may interfere between themselves, this term accounts for that interference. When in a single BS operation, account the power between BS $b$, UE $m$, in layer $l'$.
        \item Compute Intra-cell interference $P_\text{IntraCI}$ - sum the interference power contributions from transmissions between UEs $m'$ and the serving-BS $b$, from all layers, scheduled between $m'$ and $b$.
        \item Compute Inter-cell interference $P_\text{InterCI}$ - sum the interference power contributions from transmissions between others UEs $m'$ and other BSs $b'$, from all layers.
        \item Compute Noise - The noise power $P_N$ is equated considering thermal noise at temperature $T$, bandwidth $B$ over which the power is spread out and a Boltzmann constant $k_B = 1.380649 \times 10^{-23}$ J/K (see Equation \ref{eq:noise}).
        
        \begin{equation} \label{eq:noise}
            P_N = k_B T B
        \end{equation}

    \end{enumerate}

    And then bring together the terms in the SINR expression in Equation \eqref{eq:simple_SINR}. Refer to Appendix \ref{ap:b} for an integral derivation of the SINR framework used in this thesis, as well as the explicit equations used for each of the quantities mentioned above. 

    \begin{equation} \label{eq:simple_SINR}
        SINR_i = \frac{P_s}{P_\text{ILI} + P_\text{IntraCI} + P_\text{InterCI} + P_N}
    \end{equation}

    Note that the correct transmit power to use is the transmit power per PRB $P_\text{PRB}$ which comes from dividing the total transmission power $P_t$ by the number of scheduled PRBs $N_{bml}^\text{PRB}$.
    At last, each PRB's SINR is divided (or subtracted when computing in dBs) by the noise figures $NF_\text{BS}$ or $NF_\text{UE}$ which are applied only at the reception to account for all hardware imperfections.

    \item Aggregate/compress the vector of realised SINRs (one SINR per PRB) into a single effective SINR for the transmission - this is done with \ac{MIESM} suggested by \cite{1696622} and \cite{1656798}. We choose this SINR mapping strategy because \cite{1656798, 4657235, 5982870, 6008103, miesm1} make a compelling case in favour of this mapping method, showing that it unquestionably achieves the very good results without the need of calibration for different MCSs. We confirmed the results and present a clear description of how it works in Appendix \cite{ap:c}. 

    \item Compute \ac{BLER} from $SINR_\text{eff}$ and the MCS used for the transmission by using curve with the correspondent CQI index in Figure \ref{fig:blercurves}.
    
    \item Flip a \acs{BLER}-biased coin to determine if each block was well received or not.
    
    \item Update the \acs{OLLA} parameter based on success or not from the block transmission - the OLLA parameter is updated when \ac{TB} has errors or when a TB is successfully transmitted. Equations \eqref{eq:ola_update1_success} and \eqref{eq:ola_update1_fail} show the how $\Delta_{OLLA}$ is updated when the block is successfully transmitted and when the block has errors, respectively. Observe the subtlety of the asymmetry in update. The term that multiplies the step size $\gamma_{OLLA}$ is much bigger in \eqref{eq:ola_update1_fail} than in \eqref{eq:ola_update1_success}, since $BLER_0$ is usually 0.1 or smaller. It's a defensive approach, to take bigger steps when there are errors: because it is always better to have some bitrate than no bitrate.
    
    \todo[inline]{This approach is too greedy when all information goes in one transport block, because it increases the MCS until there's an error. We have to have several transport blocks in order to avoid converging to a no-bitrate TTI. With more transport blocks, it's possible to have some errors but still a quite good bitrate. The step-size should accout for how many transport blocks we have since we are updating the OLLA param per transport block..}

    \begin{equation} \label{eq:ola_update1_success}
        \Delta_{OLLA} = \Delta_{OLLA} + BLER_0 \times \gamma_{OLLA}
    \end{equation}

    \begin{equation} \label{eq:ola_update1_fail}
        \Delta_{OLLA} = \Delta_{OLLA} - (1 - BLER_0) \times \gamma_{OLLA}
    \end{equation}
\end{enumerate}




As a final step to simulate all processes that happen during a TTI, we need to update the buffers with the information that was transmitted. We model an ordered buffer, i.e. if only third transport block in a set of five block gets lost, then the bits that would be sent in that transport block are not removed from the buffer.
In essence, this means that the BLER may cause packets to be arrive out of order. This phenomenon is represented in Figure \ref{fig:buf}: assuming a TB with the same size as a packet, the bits that didn't arrive successfully are kept in the transmission buffer.

\image{SLS/ordered_buffer.png}{Buffer state before and after a transmission, with $\ac{TBS} = L_\text{pck}$}{fig:buf}{.4}


Further note that if a TB is not the size of a packet, which happens more frequently than otherwise, and gets lost, then the bits from more than one packet will be left waiting at the transmission buffer. This increases the likelihood that a packet is discarded due to excessive delay, but makes the modelling considerable more realistic. So, the latencies supported in a given scenario are closer to reality, which, unfortunately, means higher latencies.

Following the buffers update, all processing for one TTI is finished and next TTI can be processed.






\begin{comment}
Finally, the changes that need to occur to process UL TTIs are below:
    \begin{itemize}
        \item In the step power control (the seventh step) there needs to be additionally a resource attribution process - instead of wideband scheduling, there must be a more granular distribution of the time-frequency resources because the point of reception (the BS) is may now receive transmissions that may interfere amongst themselves. The \acsp{PRB} attributed per UE depend on the channel quality of the UE. 
        
        \item The noise computation in the SINR estimation steps is going to be slightly different because the used bandwidth is different - for the first estimation, it is enough to consider the same bandwidth for all users. For step eight, the last estimation, a new and more precise value can be computed for the total noise power. 
        
        % Check this crap.
        Moreover, since interference is expected to be far more relevant than noise, the wideband noise can be considered for both cases, giving a marginally different estimate, even though it is a worst case scenario estimate noise-wise.
        

        \item The interference needs to be computed differently for the uplink. The inter-cell and intra-cell interference are now replaced by UE transmissions to the respective BSs. 
        
        
        \item The transmit power in the UL is always the same, it may be more or less distributed across frequency. Unless the in SINR is limited by too much interference and using the less power results in similar SINRs, which is rarely the case, the transmit power can be always maximum.
    \end{itemize}

\end{comment}
