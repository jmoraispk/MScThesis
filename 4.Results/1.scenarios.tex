\section{Scenarios}
\label{sec:scenarios}

This chapter presents the results that validate the modelling framework. In particular, this section defines the scenarios concretely, providing values to the parameters defined in the (previous) modelling chapter. 

Firstly, we analyse the case with only one user. In this case, there is no interference thus the \acs{SINR} is very high. Still with one user, we obtain the maximum achievable bit rate, which requires us to change the average DL application-layer bit rate to a much higher value to emulate an always-full buffer. 

Posteriorly, we simulate four physically present users. In doing so, we see how the received signal strength varies across time non-identically for different users, how the link adaptation parameter \acs{OLLA} changes depending on block errors, and correlate different metrics with one another, proving the consistency of the model.

\section*{Parameters}

With the exception of the number of physical users present in the meeting room $N_{phy}$ and the average downlink application bit rate $\overline{R}_{DL}$, there are no changes in the parameters throughout the chapter. See the values of the all necessary parameter below: Table \ref{tab:app_param} contains parameters for the application layer and Table \ref{tab:radio_param} contains parameters from the radio layer. 

The parameter $N_{phy}$ changes from the first section to the second, respectively, from $N_{phy} = 1$ to $N_{phy} = 4$. The parameter $\overline{R}_{DL}$ is always 80 Mbps with the exception of one plot, Figure \ref{fig:su_bitrate_full}, where it is put to 500 Mbps. 

Most parameters have the values from the examples provided when introducing the parameter in the previous Chapter. Some parameters without examples were, e.g. the number of slots between CSI anc scheduling updates, respectively, $N_{slots}^{CSI}$ $N_{slots}^{SCH}$. We attempt to make the case-study as simple as possible, so we both to be unitary. For the same reason, we set the acknowledgement delay $\tau_{ACK}$ to zero. This means the data that has errors in one slot can be sent again in the next. 

Like in the modelling, UL is disabled. Therefore, there are no UL slots, resulting in $s_{TDD} = 0$. Furthermore, a heavy slot format is considered, where all symbols are of the same type as the slot, DL in this case. However, we do not take into account signalling and slot-format overheads, i.e. $\eta_{OH} = \eta_{slot} = 1$. Thus, we compute a radio-layer bit rate, not an application-layer throughput. 

We perform a simulation for the duration $T_{sim}$ of 1 second, such that we can see more precisely the variations and accurately make correlations between variables. We use numerology two because it is the only numerology present in sub-6 GHz and in mmWAves. With $\mu=2$, one second corresponds to 4000 TTIs. Therefore whenever a plot shows time on the horizontal axis, each second holds 4000 values.

Some final considerations and clarifications regarding the choice of parameters:

\begin{itemize}
    \item $N_{vir}$ must change with $N_{phy}$ to keep eight users around the table such that each present user has the same head movement patterns in both scenarios;
    \item $\mathcal{F}$ represents the set of simulated frequencies. We simulate both 3.5 and 26 GHz. Since every conclusion and relation between parameters apply to any frequency, the results for 26 GHz are in Appendix \ref{ap:mmWave_simulation_plots}, due to space constraints;
    \item We use $P_{t,max}^{BS} = 0.1 $ Watt \cite{etsi_wifi_tx_power}, which is the same value used in Wi-Fi access points in the 2.4 GHz band, thus it is a safe and conservative approach.
\end{itemize}



\begin{table}[h]
    \centering
    \caption{Application layer parameters.}
    \label{tab:app_param}
    \begin{tabular}{|c|c|c|c|c|c|}
        \hline
        Variable    & Value         & Variable   & Value       & Variable            & Value   \\ \hline
        $N_{phy}$ & 1 or 4 & $d_{u}$    & 0.05 m  & $N_{ant}^{UE}$ & \begin{tabular}[c]{@{}c@{}}3.5 GHz:  2x2\\ 26 GHz: 8x8\end{tabular}       \\ \hline
        $N_{vir}$ & 7 or 4 & $\sigma_x$ & 0.667 m & $N_{ant}^{BS}$ & \begin{tabular}[c]{@{}c@{}}3.5 GHz:  4 x 4\\ 26 GHz: 16 x 16\end{tabular} \\ \hline
        $N_{cam}$   & 0             & $\sigma_y$ & 0.667 m     & $\overline{R}_{DL}$ & 80 Mbps \\ \hline
        $S_{room}$  & {[}6,6,3{]} m & $\sigma_z$ & 0.223 m     & $S_{GoP}$           & 6       \\ \hline
        $h_{user}$  & 1.4 m         & $\beta_x$  & $\pi/9$ rad & $r_{P/I}$           & 0.2     \\ \hline
        $h_{table}$ & 1 m           & $\beta_y$  & $\pi/6$ rad & $R_F$               & 30      \\ \hline
        $d_u$       & 0.6 m         & $\beta_z$  & $\pi/9$ rad & $\delta$            & 3       \\ \hline
        $d_s$       & 0.3 m         & $r_t$      & 1 m         & $\gamma$            & 0.5      \\ \hline
        $d_{o}$     & 0.15 m        & $r_u$      & 1.2 m       & $o$                 & 0        \\ \hline
    \end{tabular}
\end{table}


\begin{table}[h]
    \centering
    \caption{Radio layer parameters.}
    \label{tab:radio_param}
    \begin{tabular}{|c|c|c|c|c|c|}
        \hline
        Variable      & Value         & Variable          & Value & Variable   & Value \\ \hline
        $T_{sim}$     & 1 s           & $P_{t,max}^{BS}$  & 0.1 W & $NF_{UE}$  & 8 dB  \\ \hline
        $\mathcal{F}$ & {3.5, 26} GHz & $N_{slots}^{CSI}$ & 1     & T          & 290 K \\ \hline
        $\mu$         & 2             & $N_{slots}^{SCH}$ & 1     & $a_\phi$   & -60º  \\ \hline
        B             & 40 MHz        & $N_{CSI}$         & 1     & $a_\theta$ & -60º  \\ \hline
        $N_{PBR}$     & 50            & $\kappa$          & 1     & $b_\phi$   & 60º   \\ \hline
        $N_{BS}$      & 1             & $BLER_0$          & 0.1   & $b_\theta$ & 60º   \\ \hline
        $s_{TDD}$    & 0 & $\gamma_{OLLA}$ & 0.1  & $r_\phi$   & \begin{tabular}[c]{@{}c@{}}3.5 GHz: 30º \\ 26GHz: 4º\end{tabular} \\ \hline
        $\tau_{ACK}$ & 0 & $t_w$           & 8000 & $r_\theta$ & \begin{tabular}[c]{@{}c@{}}3.5 GHz: 30º\\ 26GHz: 4º\end{tabular}  \\ \hline
        $\tau_{CSI}$  & 4             & $N_{TB}$          & 5     & $\eta$     & 0     \\ \hline
    \end{tabular}
\end{table}





 